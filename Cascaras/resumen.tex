%---------------------------------------------------------------------
%
%                      resumen.tex
%
%---------------------------------------------------------------------
%
% Contiene el cap�tulo del resumen.
%
% Se crea como un cap�tulo sin numeraci�n.
%
%---------------------------------------------------------------------

\chapter{Resumen}
\cabeceraEspecial{Resumen}

En este proyecto se estudia y desarrolla una herramienta basada en depuraci�n simb�lica que permite conocer el funcionamiento de programas inform�ticos sin ser ejecutados en base a observar todas sus posibles ramas de ejecuci�n (hasta un cierto nivel), as� como los correspondientes pares entrada-salida. Dicha herramienta est� ideada con la idea de ayudar a los estudiantes de iniciaci�n a la programaci�n, y en general a programadores inexpertos a la hora de razonar acerca de la correcci�n de sus programas.

El n�cleo de la herramienta est� basado en la ejecuci�n simb�lica, una de las t�cnicas m�s potentes para hacer estudios est�ticos del comportamiento de los programas. La ejecuci�n simb�lica consiste en el an�lisis de un programa usando valores simb�licos (o variables) en lugar de valores concretos. Esto permite comprender la ejecuci�n del programa gracias a la observaci�n de los comportamientos de sus diferentes caminos de ejecuci�n, as� como determinar las condiciones que deben ser verificadas por los datos de entrada para obtener un resultado particular, y la relaci�n entre los valores ingresados y producidos en la ejecuci�n de un programa.

La herramienta desarrollada, denominada SymC++, recibe una serie de par�metros de entrada sobre la funci�n a testear, partiendo de un c�digo C++, obteniendo as� informaci�n de las distintas ramas generadas. Dichos  resultados se generan en formato XML lo cual facilita la portabilidad y escalabilidad del proyecto ofreciendo un depurador que se puede adaptar a diferentes interfaces de usuario.


%-------------------------------------------------------------------
\section*{Palabras clave}
%-------------------------------------------------------------------
\label{ap1:palClave}

Ejecuci�n simb�lica, ramas de ejecuci�n, C++, XML, depuraci�n, valor simb�lico.

\endinput
% Variable local para emacs, para  que encuentre el fichero maestro de
% compilaci�n y funcionen mejor algunas teclas r�pidas de AucTeX
%%%
%%% Local Variables:
%%% mode: latex
%%% TeX-master: "../Tesis.tex"
%%% End:
