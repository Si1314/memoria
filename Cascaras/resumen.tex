%---------------------------------------------------------------------
%
%                      resumen.tex
%
%---------------------------------------------------------------------
%
% Contiene el cap�tulo del resumen.
%
% Se crea como un cap�tulo sin numeraci�n.
%
%---------------------------------------------------------------------

\chapter{Resumen}
\cabeceraEspecial{Resumen}

En este proyecto se estudia y desarrolla una herramienta de ``depuraci\'on simb\'olica'' que permite conocer el funcionamiento de programas inform\'aticos sin ser ejecutados, en base a observar todas sus posibles ramas de ejecuci\'on (hasta un cierto nivel) as\'i como los correspondientes pares entrada-salida. Dicha herramienta podr\'a ayudar enormemente a los estudiantes de iniciaci\'on a la programaci\'on, y en general a programadores inexpertos, a la hora de razonar acerca de la correcci\'on de sus programas.

Para ello haremos uso de la ejecuci�n simb�lica, esta herramienta es una de las t�cnicas m�s potentes para hacer razonamientos est�ticos acerca del comportamiento de los programas. Se basa en la ejecuci�n de un programa usando valores simb�licos (o variables) en lugar de valores concretos. Esto permite hacer razonamientos sobre programas en base a observar los comportamientos de sus diferentes caminos de ejecuci�n as� como determinar las condiciones que deben ser verificadas por los datos de ingreso para que un camino particular se ejecute, y la relaci�n entre los valores ingresados y producidos en la ejecuci�n de un programa. 

Durante este proyecto hemos desarrollado una herramienta denominada {\it SymC++} basada en la ejecuci�n simb�lica para la generaci�n de test de programas escritos en C++. Esta herramienta necesitar� una serie de par�metros de entrada sobre la funci�n a testear y as� obtener informaci�n de las diferentes ramas generadas. Por �ltimo lado, todos los resultados se generar�n en formato XML lo cual facilita la portabilidad y depuraci�n sin necesidad de la interfaz de usuario.

%-------------------------------------------------------------------
\section*{Palabras clave}
%-------------------------------------------------------------------
\label{ap1:palClave}

Ejecuci�n simb�lica, ramas de ejecuci�nd, C++, XML, depuraci�n, valor simb�lico.

\endinput
% Variable local para emacs, para  que encuentre el fichero maestro de
% compilaci�n y funcionen mejor algunas teclas r�pidas de AucTeX
%%%
%%% Local Variables:
%%% mode: latex
%%% TeX-master: "../Tesis.tex"
%%% End:
