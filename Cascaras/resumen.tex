%---------------------------------------------------------------------
%
%                      resumen.tex
%
%---------------------------------------------------------------------
%
% Contiene el cap�tulo del resumen.
%
% Se crea como un cap�tulo sin numeraci�n.
%
%---------------------------------------------------------------------

\chapter{Resumen}
\cabeceraEspecial{Resumen}

La ejecuci�n simb�lica o evaluaci�n simb�lica es un modo de analizar programas para determinar qu� entradas causan cada parte del programa a ejecutar. En este tipo de programas, un int�rprete recorre el programa donde cada variable asume un valor simb�lico acotado. Este recorrido dar� como resultado expresiones en t�rminos de dichos s�mbolos de expresiones y variables y las limitaciones en cuanto a los s�mbolos de los posibles resultados de cada rama condicional. 

Dicho recorrido nos permite determinar las condiciones que deben ser verificadas por los datos de ingreso para que un camino particular se ejecute, y la relaci�n entre los valores ingresados y producidos en la ejecuci�n de un programa. 

Durante este proyecto hemos desarrollado una herramienta llamada \\SymC++ que se basa en la ejecuci�n simb�lica para generar test en programas escritos en C++, escribiendo como par�metros el nombre del m�todo, el intervalo de n�meros enteros y la profundidad de los bucles, para ejecutar la funci�n simb�licamente y obtener informaci�n de las diferentes ramas generadas. Por otro lado, se podr� obtener informaci�n a trav�s de los XMLs generados por el �rbol de ejecuci�n o los resultados obtenidos.

El objetivo del proyecto es dotar a los estudiantes de iniciaci�n a la programaci�n de una herramienta que les permita depurar sus programas por un m�todo diferente a la prueba y error.

%-------------------------------------------------------------------
\section*{Palabras clave}
%-------------------------------------------------------------------
\label{ap1:palClave}

Ejecuci�n simb�lica, Evaluaci�n simb�lica, C++, XML, depuraci�n, int�rprete, valor simb�lico.

\endinput
% Variable local para emacs, para  que encuentre el fichero maestro de
% compilaci�n y funcionen mejor algunas teclas r�pidas de AucTeX
%%%
%%% Local Variables:
%%% mode: latex
%%% TeX-master: "../Tesis.tex"
%%% End:
