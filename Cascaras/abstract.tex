%---------------------------------------------------------------------
%
%                      resumen.tex
%
%---------------------------------------------------------------------
%
% Contiene el cap�tulo del resumen.
%
% Se crea como un cap�tulo sin numeraci�n.
%
%---------------------------------------------------------------------

\chapter{Abstract}
\cabeceraEspecial{Abstract}

This project studies and develops a tool based on symbolic debugging that allows reasoning about programs behavior statically, i.e., without being executed, by means of observing all possible branches of execution (up to a level) and their corresponding input-output behaviors. The tool is designed with the idea of helping students of introduction to programming, and, in general,  inexperienced developers to reason about the correctness of their programs.

The core of the tool is based on symbolic execution, one of the most powerful technologies for statically studying programs behavior. Symbolic execution consists in executing a program using symbolic values (or variables) instead of concrete values. This allows understanding the program execution by means of observing the behavior of the different execution branches and to determine the conditions that must be verified by the input data to obtain a particular result, and the relationship between the input and produced values in the execution of the program.

The developed tool, called SymC++, receives a set of input parameters of the function to be tested (a C++ function), and obtains information about the generated execution branches. The results are output in XML format, which facilitates portability and scalability, hence providing a debugger that can be adapted to different user interfaces.

%-------------------------------------------------------------------
\section*{Key words}
%-------------------------------------------------------------------
\label{ap1:keyWord}

Symbolic execution, Symbolic evaluation, C++, XML, symbolic value

\endinput
% Variable local para emacs, para  que encuentre el fichero maestro de
% compilaci�n y funcionen mejor algunas teclas r�pidas de AucTeX
%%%
%%% Local Variables:
%%% mode: latex
%%% TeX-master: "../Tesis.tex"
%%% End:
