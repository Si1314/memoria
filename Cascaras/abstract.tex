%---------------------------------------------------------------------
%
%                      resumen.tex
%
%---------------------------------------------------------------------
%
% Contiene el cap�tulo del resumen.
%
% Se crea como un cap�tulo sin numeraci�n.
%
%---------------------------------------------------------------------

\chapter{Abstract}
\cabeceraEspecial{Abstract}


Symbolic execution or symbolic evaluation is a means of analyzing programs to determine what inputs cause each part of a program to execute. In this kind of programs, an interpreter follows the program where every variable assume a bounded symbolic value. It thus arrives expressions in terms of those symbols for expressions or variables, and constraints in terms of those symbols for the possible outcomes of each conditional branch.

This path allows us to determine the conditions that have to be verified by the input data for a particularly branch is executed, and the relationship between input values and the produced by program execution.

During this project we have developed a tool called SymC++ which is based on symbolic execution to generate test programs written on C++. We have to write like parameters the method name, the range of integers and the deep to bucles, to execute it symbolically and get the information generated from different branches. On the other side, we could get information through the XMLs generated by the execution tree or results.

The project objective is to provide students of programming introduction a tool that allows them to debug their programs by a different method of trial and error.

%-------------------------------------------------------------------
\section*{Key words}
%-------------------------------------------------------------------
\label{ap1:keyWord}

Symbolic execution, Symbolic evaluation, C++, XML, symbolic value

\endinput
% Variable local para emacs, para  que encuentre el fichero maestro de
% compilaci�n y funcionen mejor algunas teclas r�pidas de AucTeX
%%%
%%% Local Variables:
%%% mode: latex
%%% TeX-master: "../Tesis.tex"
%%% End:
