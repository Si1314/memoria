%---------------------------------------------------------------------
%
%                          Cap\'itulo 3
%
%---------------------------------------------------------------------

\chapter{Int\'erprete con restricciones}

\pagestyle{plain}

En este cap\'itulo hablaremos sobre la parte principal de nuestro proyecto, el Int\'erprete Simb\'olico, que recibe el \'arbol sint\'actico generado por la herramienta AST2XML, realiza una ejecuci\'on simb\'olica de \'este devolviendo un conjunto de posibles valores de entrada con los correspondientes valores de salida generados.

%-------------------------------------------------------------------
\section{El lenguaje Prolog y la librer\'ia clpfd}
%-------------------------------------------------------------------

La Programaci\'on L\'ogica tiene sus or\'igenes m\'as cercanos en los trabajos de prueba autom\'atica de teoremas de los a�os sesenta. J. A. Robinson propone en 1965 una regla de inferencia a la que llama resoluci\'on, mediante la cual la demostraci\'on de un teorema puede ser llevada a cabo de manera autom\'atica. La resoluci\'on es una regla que se aplica sobre cierto tipo de f\'ormulas del C\'alculo de Predicados de Primer Orden, llamadas cl\'ausulas y la demostraci\'on de teoremas bajo esta regla de inferencia se lleva a cabo por reducci\'on al absurdo. La realizaci\'on del paradigma de la programaci\'on l\'ogica es el lenguaje Prolog.

Prolog es un lenguaje de programaci\'on ideado a principios de los a�os 70 en la Universidad de Aix-Marseille I (Marsella, Francia). No tiene como objetivo la traducci\'on de un lenguaje de programaci\'on, sino la clasificaci\'on algor\'itmica de lenguajes naturales. [27]

La idea es especificar formalmente los enunciados utilizando la l\'ogica de predicados de manera que Prolog sea capaz de interpretar e inferir soluciones a partir de esos enunciados.

La l\'ogica de predicados estudia las frases declarativas teniendo en cuenta la estructura interna de las proposiciones, es decir, un conjunto de cl\'ausulas de la forma: ``q :- p'', en la que si p es cierto entonces q es cierto. Una cl\'ausula pueden ser un conjunto de hechos positivos, por ejemplo de la forma ``q :- p, r, s.'' ; una implicaci\'on con un \'unico consecuente, ``q :- p'' ; un hecho positivo, ``p.'' ; instrucciones con par\'ametros de Entrada/Salida ``p(Entrada,Salida).'' ; o incluso llamadas a otras funciones.

En Prolog, los predicados se contrastan en orden, la ejecuci\'on se basa en dos conceptos: la unificaci\'on y el backtracking. Una vez ejecutamos una funci\'on de Prolog se sigue ejecutando el programa gracias a las llamadas a predicados, si procede, hasta determinar si el objetivo es verdadero o falso. Todos los objetivos terminan su ejecuci\'on en \'exito (``verdadero''), o en fracaso (``falso''). Si el resultado es falso entra en juego el backtracking, es decir, deshace todo lo ejecutado situando el programa en el mismo estado en el que estaba justo antes de llegar al punto de elecci\'on, ah\'i se toma el siguiente punto de elecci\'on que estaba pendiente y se repite de nuevo el proceso, de ah\'i la utilidad de prolog en nuestro proyecto ya que nos permite implementar un int\'erprete de ejecuci\'on simb\'olica de forma natural y pr\'acticamente nativa.

La Programaci\'on por restricciones es un paradigma de la programaci\'on en inform\'atica donde las variables est\'an relacionadas mediante restricciones (ecuaciones). Se emplea en la descripci\'on y resoluci\'on de problemas combinatorios, especialmente en las \'areas de planificaci\'on y programaci\'on de tareas (calendarizaci\'on). La programaci\'on con restricciones se basa principalmente en buscar un estado en el cual una gran cantidad de restricciones sean satisfechas simult\'aneamente, expres\'andose un problema como un conjunto de restricciones iniciales a partir de las cuales el sistema construye las relaciones que expresan una soluci\'on.

Como ya se introdujo en el estado del arte, la programaci\'on con restricciones proporciona una herramienta en la que las variables est\'an expresadas en t\'erminos de restricciones o ecuaciones. Para el desarrollo de esta herramienta hemos necesitado hacer uso de este paradigma de programaci\'on, el cual nos facilitaba mostrar restricciones como datos de salida.

La librer\'ia clpfd ``Constraint Logic Programming over Finite Domains'' (Programaci\'on l\'ogica basada en restricciones sobre dominios finitos) proporciona una aritm\'etica completa para variables restringidas a valores enteros o elementos at\'omicos, se puede utilizar para modelar y resolver diversos problemas combinatorios tales como las tareas de planificaci\'on, programaci\'on y asignaci\'on. [28]

Para el desarrollo de nuestro int\'erprete, como ya indicamos anteriormente, optamos por el  lenguaje de programaci\'on l\'ogica Prolog, en gran parte debido a la familiaridad que nos supone, y al conjunto de facilidades relacionadas con el backtracking y la programaci\'on por restricciones que ya conoc\'iamos de antemano.

%-------------------------------------------------------------------
\section{Dise�o e interfaz del int\'erprete}
%-------------------------------------------------------------------

Como ya hemos indicado antes, AST2XML devuelve un archivo en formato XML con una versi\'on simplificada del \'arbol sint\'actico que emplea Clang con el que realizaremos la ejecuci\'on simb\'olica. Para llevar a cabo una ejecuci\'on simb\'olica es necesario tener un sistema capaz de establecer restricciones l\'ogicas y matem\'aticas y que aparte las pueda resolver.

El int\'erprete est\'a encapsulado en dos m\'odulos para hacer m\'as f\'acil su dise�o: {\it VariablesTable.pl} e {\it Interpreter.pl}. 

%-------------------------------------------------------------------
\subsection*{VariablesTable.pl}
%-------------------------------------------------------------------

El int\'erprete ir\'a guardando las variables y su respectivo valor en una tabla de variables representado en prolog mediante una lista, de forma que todas las posibles operaciones que se puedan aplicar sobre ella est\'en encapsuladas en un mismo m\'odulo. Por ejemplo, funciones como a�adir un elemento a la tabla (add), obtener el valor de una variable (getValue), modificar el nombre de una variable (updateNames), etc. son funciones que s\'olo se aplican sobre la tabla de variables.

%-------------------------------------------------------------------
\subsection*{Interpreter.pl}
%-------------------------------------------------------------------

Por otro lado, este m\'odulo representa el Int\'erprete que ser\'a llamado por la interfaz y el que devolver\'a la soluci\'on. Los par\'ametros de entrada son: fichero de entrada, fichero de salida, Inf, Sup, MaxDepth y nombre de funci\'on.

%-------------------------------------------------------------------
\subsubsection*{Fichero de entrada}
%-------------------------------------------------------------------

Indica el nombre del fichero .xml donde est\'a el c\'odigo traducido por clang. 

\lstinputlisting[caption=Ejemplo de XML de entrada que recibe el int\'erprete, style=customxml]{Codigos/ejemploXMLClang.xml}

%-------------------------------------------------------------------
\subsubsection*{Fichero de salida}
%-------------------------------------------------------------------

Indica el nombre del fichero .xml de salida que el int\'erprete devolver\'a con las soluciones para una determinada funci\'on.

\lstinputlisting[caption=Ejemplo de XML obtenido por Clang, style=customxml]{Codigos/ejemploXMLProlog.xml}

%-------------------------------------------------------------------
\subsubsection*{Inf, Sup y MaxLoop}
%-------------------------------------------------------------------

Inf y Sup son los valores que representan el l\'imite del dominio de los valores de entrada de la funci\'on a interpretar. MaxLoop representa el valor del n\'umero m\'aximo de veces que puede ejecutarse una estructura de control: while y for.

%-------------------------------------------------------------------
\subsubsection*{Nombre de la funci\'on}
%-------------------------------------------------------------------

Para indicar el nombre de la funci\'on que vamos a querer probar entre todas las funciones posibles contenidas en el fichero pasado en ``Fichero de entrada''.

%-------------------------------------------------------------------
\section{Ciclo de ejecuci\'on}
%-------------------------------------------------------------------

Como forma explicativa del ciclo de ejecuci\'on que sigue el int\'erprete pondremos el siguiente ejemplo en el de devolvemos el resultado obtenido por los dos par\'ametros de entrada:

\lstinputlisting[caption=Ejemplo ilustrativo con funci\'on suma, style=customcpp]{Codigos/ejemploCicloCC.cpp}

Como ya sabemos, el int\'erprete recibir\'a un fichero XML como par\'ametro de entrada, junto con otros par\'ametros que explicaremos m\'as adelante. El c\'odigo de nuestro ejemplo tendr\'a el siguiente aspecto una vez pasado por AST2XML:

\lstinputlisting[caption=\'arbol sint\'actico de la funci\'on suma, style=customxml]{Codigos/ejemploCicloXML.xml}

El primer paso es la conversi\'on de la estructura XML en una lista de elementos mediante la funci\'on load\_xml\_file(+File,-DOM) que nos aporta la librer\'ia SGML [29]. Para nuestro ejemplo, le pasaremos por el par\'ametro File el XML con el c\'odigo anterior y nos devolver\'a en el par\'ametro de salida DOM una lista de elementos de la forma:

\lstinputlisting[caption=Elementos correspondientes a la funci\'on suma, style=customxml]{Codigos/ejemploCicloElementos.pl}

Cada element tiene tres argumentos: el nombre del nodo ({\it nombreNodo}), los atributos del nodo ({\it atributosNodo}) y el cuerpo del nodo ({\it cuerpoNodo}).

\begin{center}
\textbf{<nombreNodo atributosNodo>\\
// cuerpoNodo\\
</nombreNodo>}\\
\end{center}

Otro par\'ametro que recibe el int\'erprete es el nombre de la funci\'on que queremos probar, tenemos que indicarlo puesto que en un mismo XML podremos tener m\'as de una funci\'on. De esta forma buscaremos previamente la funci\'on que el usuario nos haya indicado e iremos pasando por cada una de las instrucciones del c\'odigo de dicha funci\'on y ejecut\'andolas mediante las sucesivas llamadas a la funci\'on execute que se detallar\'a m\'as adelante.

El resto de par\'ametros que el int\'erprete recibir\'a son:

\begin{itemize}
\item \textbf{Outfile}: para indicar el nombre del archivo en el que se guardar\'a el resultado de la ejecuci\'on, tanto las posibles entradas como sus respectivas salidas generadas.
\item \textbf{Inf, Sup}: par\'ametros que nos servir\'a para delimitar el valor posible de las salidas.
\item \textbf{MaxDepth}: para poner un n\'umero m\'aximo de vueltas que puede llegar a hacer un bucle si la condici\'on sigue siendo cierta en el momento en el que se alcanza dicho valor.
\end{itemize}

%-------------------------------------------------------------------
\subsection*{execute/3}
%-------------------------------------------------------------------

Es un predicado encargado de ejecutar una instrucci\'on. Sus argumentos son: el estado previo a la ejecuci\'on de unas instrucciones, la lista que contiene las instrucciones y el estado posterior a la ejecuci\'on. Este predicado se plante\'o de forma que pudi\'eramos controlar cada una de las posibles instrucciones que puede llegar a haber en el c\'odigo. Para conseguirlo en cada una de estas posibilidades expresamos expl\'icitamente los casos de forma que la instrucci\'on que se ejecuta en un momento dado es el primer elemento de la lista de instrucciones que recibe la funci\'on. Una vez se hayan completado los pasos que simulan el comportamiento de dicha instrucci\'on, se hace una ``llamada'' recursiva a execute con el estado resultante de la ejecuci\'on, el resto de la lista de instrucciones y el estado posterior.

Esta forma de realizar los pasos de c\'omputo se asemeja, en un aspecto m\'as te\'orico, a la sem\'antica de paso corto (sem\'antica operacional), aunque dicha similitud no es completa puesto que al trabajar con instrucciones puntuales llevamos a cabo muchos pasos intermedios.

Es importante destacar que el bactracking inherente de prolog se restringe en el predicado {\it execute}. Esto se debe a que el orden de las instrucciones es siempre el mismo. En el int\'erprete hemos limitado el backtracking a la evaluaci\'on de condiciones puesto que \'estas son las que realmente definen el flujo del programa. Es por ello que, en las instrucciones de control, llegamos a necesitar la presencia de otros predicados que, adem\'as de dirigir el comportamiento de la ejecuci\'on, permiten limitar el n\'umero de repeticiones que realiza un bucle.

Iremos almacenando en cada paso el estado actual donde nos encontramos con la siguiente informaci\'on:

\begin{itemize}
\item \textbf{Table}.\\
Llevamos la tabla de variables correspondientes al estado en el que nos encontramos. En esta tabla contemplamos la posiblidad de estructurarla en \'ambitos de forma que el acceso a las variables sea consistente en cada momento.
\item \textbf{Cin, Cout} \\
Listas con los valores de entrada (Cin) o salida (Cout) si procede.
\item \textbf{Trace} \\
Lista que conserva el n\'umero de las l\'ineas de las instrucciones por las que hemos pasado para poder m\'as tarde marcarlas en la interfaz.
\end{itemize}

%-------------------------------------------------------------------
\section{Funciones, declaraci\'on y asignaci\'on de variables.}
%-------------------------------------------------------------------

%-------------------------------------------------------------------
\subsection*{Definici\'on de una funci\'on}
%-------------------------------------------------------------------

Una funci\'on se define siguiendo el siguiente esquema:

\begin{center}
\textbf{<tipo\_valor\_retorno> <nombreFuncion>(<lista\_par\'ametros>){ [sentencias] }}
\end{center}

\lstinputlisting[caption=Instrucci\'on para la definici\'on de una funci\'on, style=custompl, firstline=1, lastline=1]{Codigos/codigosProlog.pl}

Ya sea el caso de la funci\'on principal o una llamada a una funci\'on esta instrucci\'on se encargar\'a de ejecutar el cuerpo de la funci\'on y despu\'es el resto de las instrucciones. Diferenciamos tambi\'en si se trata de una funci\'on o procedimiento ya que en el caso de las funciones tiene que existir un valor de retorno, al contrario que los procedimientos. Como vemos en el esquema anterior, la funci\'on puede llegar a contener \textbf{par\'ametros} de entrada que leeremos mediante la instrucci\'on:

\lstinputlisting[caption=Intrucci\'on que recibe los par\'ametros de entrada, style=custompl, firstline=3, lastline=3]{Codigos/codigosProlog.pl}

Con esta instrucci\'on insertaremos en la tabla de variables cada uno de los par\'ametros que formen parte de la variable Parametros. Una vez actualizada la tabla llamaremos a ejecutar al resto de las instrucciones para continuar con el programa.

Un \textbf{bloque} es un mero conjunto de sentencias, que puede estar vac\'io, o encerradas por llaves {}, ese conjunto lo recogemos con la instrucci\'on correspondiente:

\lstinputlisting[caption=Intrucci\'on para el cuerpo de un bloque, style=custompl, firstline=5, lastline=5]{Codigos/codigosProlog.pl}

Este es el caso en el que ejecutamos cualquier tipo de bloque, ya sea el cuerpo de una funci\'on, if, for, while, etc. Llamamos a ejecutar al cuerpo de forma que creamos a su vez una nueva lista de variables para englobar las variables pertenecientes a ese \'ambito.

Para resolver el caso de una \textbf{llamada a funci\'on} el int\'erprete har\'a uso del precicado {\it callFunction}, que buscar\'a la funci\'on a la que se est\'a referenciando y la ejecutaremos de forma que consigamos el valor devuelto por la funci\'on como resultado de la expresi\'on.

\begin{center}
\textbf{[tipo] <variable> = <nombreFuncion>(<lista\_par\'ametros>) ;}
\end{center}

El tipo: [tipo] ser\'a vac\'io en el caso en el que la variable ya hubiese sido declarada con anterioridad.

%-------------------------------------------------------------------
\subsection*{Declaraci\'on de una variable}
%-------------------------------------------------------------------

A la hora de declarar una variable primero se especifica el tipo y a luego una lista de variables seguidas de un punto:

\begin{center}
\textbf{<tipo> <lista de variables> ;}
\end{center}

La lista de variables tiene que estar formada como m\'inimo por una variable. Nunca podr\'a ser una lista vac\'ia. El tipo ser\'a siempre int ya que es el tipo de variable que contempla nuestra herramienta. Ya que se pueden declarar una o varias variables en la misma instrucci\'on, diferenciamos la posibilidad de guardar en la tabla de variables una sola variable:

\lstinputlisting[caption=Instrucci\'on para la declaracion de una variable, style=custompl, firstline=9, lastline=9]{Codigos/codigosProlog.pl}

y la posibilidad de guardar en la tabla de variables varias declaraciones a la vez:

\lstinputlisting[caption=Intrucci\'on para las declaraciones de varias variables en grupo, style=custompl, firstline=7, lastline=7]{Codigos/codigosProlog.pl}

En este \'ultimo caso, internamente, se llamar\'a a la funci\'on encargada de guardar una a una las variables con sus tipos asignados pero sin valor ya que se trata \'unicamente de una declaraci\'on y no de una asignaci\'on.

A continuaci\'on se seguir\'a con la ejecuci\'on del resto de las instrucciones

%-------------------------------------------------------------------
\subsection*{Asignaci\'on de una variable}
%-------------------------------------------------------------------

Las sentencias de asignaci\'on responden al siguiente esquema:

\begin{center}
\textbf{<variable> <operador de asignaci\'on> <expresi\'on> ;}
\end{center}

\'este es el momento en el que se le da valor a una variable anteriormente declarada y por tanto ya a�adida anteriormente a la tabla de variables. Ejecutaremos el cuerpo de la asignaci\'on para obtener el valor que tomar\'a la variable y luego la actualizaremos. Las asignaciones pueden expresarse de dos formas distintas dependiendo de la forma en que se ponga el operador de asignaci\'on.

En el caso en el que el operador de asignaci\'on sea de la forma:

\begin{center}
\textbf{<variable> = <expresi\'on> ;}
\end{center}

la asignaci\'on se realizar\'a de forma que se resuelve la expresi\'on y luego se le asigna el valor resultante a la variable.

En el caso en el que el operador de asignaci\'on sea de la forma:

\begin{center}
\textbf{<variable> op= <expresi\'on>; siendo op = {+,-,*,/}}
\end{center}

\'este es un caso especial en el que la asignaci\'on en este caso no es otra cosa que hacer:

\begin{center}
\textbf{<variable> = <variable> op <expresi\'on> ;}
\end{center}

Se sigue la ejecuci\'on con la llamada de {\it execute} del resto de las instrucciones.

%-------------------------------------------------------------------
\section{Instrucciones aritm\'eticas y condicionales.}
%-------------------------------------------------------------------

Este apartado explica los predicados que se utilizar\'an a la hora de resolver expresiones aritm\'eticas y sentencia de selecci\'on if / if...else.

%-------------------------------------------------------------------
\subsection*{resolveExpression/4}
%-------------------------------------------------------------------

Este predicado servir\'a para resolver las distintas expresiones que existan. La estructura de llamada de este predicado es la que sigue:

\lstinputlisting[caption=Instrucci\'on para la resoluci\'on de las expresiones, style=custompl, firstline=39, lastline=39]{Codigos/codigosProlog.pl}

Entrada y Salida corresponden a la tabla de variables que entra y la tabla de variables actualizada despu\'es de haberse resuelto la expresi\'on. CuerpoExpresion es la expresi\'on concreta que queremos resolver y ValorRetorno el valor resultante devuelto. En este apartado nos centraremos \'unicamente en las expresiones aritm\'eticas

%-------------------------------------------------------------------
\subsection*{Operadores aritm\'eticos}
%-------------------------------------------------------------------

Los operadores se pueden clasificar atendiendo al n\'umero de operandos que afectan. Seg\'un esta clasificaci\'on tenemos tres tipos de operadores, pueden ser unitarios, binarios o ternarios. Los primeros afectan a un solo operando, los segundos a dos y los ternarios a tres.

%-------------------------------------------------------------------
\subsubsection*{Operadores unitarios}
%-------------------------------------------------------------------

\lstinputlisting[caption=Operador unario, style=custompl, firstline=15, lastline=15]{Codigos/codigosProlog.pl}

Los tipos unarios que reconoce la herramienta siguen la estructura:

\begin{itemize}
\item 
\begin{tabular}{l l}
\textbf{+<expresion>} & Para asignar valores positivos o negativos\\
\textbf{--<expresion>} & a los valores a los que se aplica.
\end{tabular}
\item
\begin{tabular}{l l}
\textbf{<variable>++} & Para incrementar o decrementar el valor \\ 
\textbf{<variable>-- --} & de la variable, ambos en una unidad.
\end{tabular}
%\textbf{+<expresion>   -<expresion>}      Para asignar valores positivos o negativos a los valores a los que se aplica
%\item \textbf{<variable>++   <variable>- -}      Para incrementar o decrementar el valor de la variable ambos en una unidad
\end{itemize}

Actualizamos el valor de la variable en la tabla de variables y continuamos con la ejecuci\'on del resto de las instrucciones.

%-------------------------------------------------------------------
\subsubsection*{Operadores binarios}
%-------------------------------------------------------------------

Los operadores de tipo binario siguen el esquema:

\begin{center}
\textbf{<variable> op <expresi\'on>; siendo op los operadores: {+,-,*,/}}
\end{center}

En este caso se resolver\'a cada expresi\'on de forma independiente mediante la llamada recursiva:

\lstinputlisting[caption=Instrucciones para la resoluci\'on de expresiones con operadores binarios, style=custompl, firstline=41, lastline=43]{Codigos/codigosProlog.pl}

Combinar\'a los dos resultados \'unicamente cuando \'estos son expresiones sencillas de forma que ya hayan llegado al caso base, que se cumple cuando se tiene que la expresi\'on es una variable o una constante de forma que ya no se pueda m\'as que operar mediante el predicado work detallado m\'as adelante.

%-------------------------------------------------------------------
\subsubsection*{Operadores l\'ogicos}
%-------------------------------------------------------------------

Nos servir\'an a la hora de definir la condici\'on de una instrucci\'on de control de flujo.

Los operadores l\'ogicos de negaci\'on siguen el esquema:

\begin{center}
\textbf{! <expresion>}
\end{center}

De forma que negar\'a el valor resultante de la expresi\'on.

Existen tambi\'en otros tipos de operadores l\'ogicos que siguen el esquema:

\begin{center}
\textbf{<expresion> op <expresion>}
\end{center}

Siendo op los valores \&\& o || ambos se resuelven en el predicado work detallado a continuaci\'on.


\lstinputlisting[caption=Asignaci\'on de una variable, style=custompl, firstline=13, lastline=13]{Codigos/codigosProlog.pl}

Este es un caso especial de la asignaci\'on ya que se trata de una asignaci\'on mediante un operador binario. Esta ejecuci\'on la haremos con la llamada a resolveExpression (explicada m\'as adelante) del cuerpo de la asignaci\'on. Como en el resto de los casos pasamos a continuaci\'on a ejecutar el resto de las instrucciones para seguir con el flujo del programa.

%-------------------------------------------------------------------
\subsection*{work/3}
%-------------------------------------------------------------------

\lstinputlisting[caption=Instrucci\'on para la ejecuci\'on de expresiones, style=custompl, firstline=45, lastline=45]{Codigos/codigosProlog.pl}

Cuya funcionalidad es \'unicamente la ejecuci\'on de la expresi\'on {\it expresi\'on1} Operador {\it expresi\'on2} dejando la soluci\'on en {\it ResultadoFinal}.

En nuestro proyecto ``Operador'' \'unicamente podr\'a ser:

\begin{table}[ht]
\begin{center}
\begin{tabular}{| l | l | l | l | l | l | l | l | l | l | l | l ||}
\hline
< & <= & > & >= & == & != & \&\& & || & + & - & * & / \\
\hline
\end{tabular}
\end{center}
\caption{Tabla con los operadores que reconoce el interprete.}
\end{table}

%-------------------------------------------------------------------
\subsection*{Instrucci\'on condicional IF, IF...ELSE}
%-------------------------------------------------------------------

Las sentencias de selecci\'on permiten controlar el flujo del programa, seleccionando distintas sentencias en funci\'on de diferentes circunstancias.

\begin{center}
\textbf{if (<condici\'on>) <sentencias>\\else <sentencias>}
\end{center}

La estructura del {\it if...else} se puede ver f\'acilmente con el diagrama que aparece en la figura 3.1.

\begin{figure}[ht]
\begin{center}
\includegraphics[width=0.9\textwidth]%
{Imagenes/Vectorial/esquemaIf}
\caption{Figura que muestra la estructura del if...else.}
\end{center}
\end{figure}

El cuerpo del if y del else tiene que estar compuesto por una sentencia como m\'inimo, en el caso de que haya dos o m\'as sentencias, \'estas estar\'an delimitadas por llaves:

\begin{center}
\textbf{\{ <sentencias> \}}
\end{center}

Se ejecutar\'a el siguiente predicado cuando se trate de una instrucci\'on condicional:

\lstinputlisting[caption=Instrucci\'on para la estructura condicional if, style=custompl, firstline=21, lastline=21]{Codigos/codigosProlog.pl}

Primero se resuelve la condici\'on y seguidamente se llamar\'a a executeBranch (detallada a continuaci\'on) con la soluci\'on obtenida, ser\'a \'esta quien distinga si la condici\'on ha salido verdadera o falsa. A continuaci\'on ejecutaremos el resto de las instrucciones para seguir con el resto del programa.

%-------------------------------------------------------------------
\subsection*{executeBranch/4}
%-------------------------------------------------------------------

Es un predicado cuya funcionalidad extiende la del execute cuando se trata del if para el caso del then y del else. Igualmente recibe un estado de entrada y devuelve otro de salida, recibe la lista de instrucciones.

\lstinputlisting[caption=Instrucci\'on para la ejecuci\'on de estructuras condicionales, style=custompl, firstline=33, lastline=33]{Codigos/codigosProlog.pl}

Aqu\'i diferenciamos los dos casos posibles: {\it Then} y {\it Else}

\begin{itemize}
\item \textbf{Caso then:}\\
Se ha cumplido la condici\'on del if por lo que la variable ResulCond pasa a valer 1 realizando por tanto la ejecuci\'on del cuerpo del then y posteriormente se llamar\'a a ejecutar al resto de las instrucciones.

\lstinputlisting[caption=Instrucci\'on para la ejecuci\'on de la rama then, style=custompl, firstline=35, lastline=35]{Codigos/codigosProlog.pl}

\item \textbf{Caso else:}\\
No se ha cumplido la condici\'on del if por lo que la variable ResulCond pasa a valer 0, en este caso realizaremos la ejecuci\'on del cuerpo del else y posteriormente se llamar\'a a ejecutar al resto de las instrucciones.

\lstinputlisting[caption=Instrucci\'on para la ejecuci\'on de la rama else, style=custompl, firstline=37, lastline=37]{Codigos/codigosProlog.pl}

\end{itemize}

%-------------------------------------------------------------------
\section{Bucles y control de terminaci\'on.}
%-------------------------------------------------------------------

Los bucles nos permiten realizar tareas repetitivas, y se usan en la resoluci\'on de la mayor parte de los problemas. En nuestro caso hablaremos \'unicamente del ``while'' y del ``for''. En ambos casos, el n\'umero de vueltas m\'aximo que pueden llegar a hacer es el marcado por el valor MaxDepth, y se har\'a uso de \'el en el caso en el que la condici\'on siga siendo cierta en el momento en el que se alcanza dicho valor, de esta forma controlaremos la terminaci\'on de los bucles.

%-------------------------------------------------------------------
\subsection*{Sentencia WHILE}
%-------------------------------------------------------------------

La sentencia {\it while} sigue el siguiente esquema:

\begin{center}
\textbf{while (<condici\'on>) <sentencia>}
\end{center}

La estructura del while se puede entender de forma gr\'afica diagrama que aparece en la figura 3.2.

\begin{figure}[ht]
\begin{center}
\includegraphics[width=0.9\textwidth]%
{Imagenes/Vectorial/esquemaWhile}
\caption{Figura que muestra la estructura del bucle while.}
\end{center}
\end{figure}

El predicado encargado de interpretar el bucle while es el que sigue:

\lstinputlisting[caption=Instrucci\'on para la ejecuci\'on del bucle while, style=custompl, firstline=23, lastline=23]{Codigos/codigosProlog.pl}

Como en cualquier instrucci\'on de flujo resolveremos la condici\'on, en el caso del while, a diferencia del if, llamaremos al predicado executeLoop (explicada m\'as adelante) con el cuerpo del while, ser\'a \'el quien distinga si la condici\'on sali\'o verdadera o falsa para ejecutar realmente el cuerpo del bucle o no. Llevaremos una variable de avance que ser\'a actualizada en el cuerpo del bucle y ser\'a quien determine cu\'ando parar. A continuaci\'on ejecutaremos el resto de las instrucciones para seguir con el resto del programa.

%-------------------------------------------------------------------
\subsection*{Sentencia FOR}
%-------------------------------------------------------------------

La sentencia {\it for} sigue el siguiente esquema:

La estructura de la sentencia for se puede entender mejor de forma gr\'afica el diagrama que aparece en la figura 3.3.

\begin{figure}[ht]
\begin{center}
\includegraphics[width=0.9\textwidth]%
{Imagenes/Vectorial/esquemaFor}
\caption{Figura que muestra la estructura del bucle for.}
\end{center}
\end{figure}

El predicado encargado de interpretar el bucle for es el que sigue:

\lstinputlisting[caption=Instrucci\'on para la ejecuci\'on del bucle for, style=custompl, firstline=25, lastline=25]{Codigos/codigosProlog.pl}

Es parecido al while ya que resolveremos la condici\'on y llamaremos al predicado executeLoop (explicada a continuaci\'on) con el cuerpo del for, ser\'a \'el quien distinga si la condici\'on sali\'o verdadera o falsa para ejecutar realmente el cuerpo del bucle o no. A continuaci\'on ejecutaremos el resto de las instrucciones para seguir con el resto del programa.

%-------------------------------------------------------------------
\subsection*{executeBranch/4}
%-------------------------------------------------------------------

El predicado executeLoop es llamado por execute cuando se encuentra una instrucci\'on de bucle while o for. Igualmente recibe un estado de entrada, la lista de instrucciones, un valor num\'erico y devuelve el estado de salida. El valor num\'erico nos servir\'a para limitar el n\'umero de vueltas que realice el bucle, de forma que no haya posibilidad de recursi\'on infinita, o un n\'umero de vueltas tan elevado que sea inviable devolver una soluci\'on. El n\'umero lo determina el usuario mediante la interfaz con valor del maxDepth como se dijo anteriormente.

Hemos limitado el backtracking a la evaluaci\'on de condiciones ya que \'estas son las que realmente definen el flujo del programa.

Se pueden diferenciar 3 tipos de executeLoop:

\begin{itemize}
\item \textbf{Caso 1:} caso en el que se llega a la \'ultima vuelta permitida por la variable maxDepth.
\lstinputlisting[caption=Instrucci\'on para el caso en el que bucle finaliza por la variable maxDepth, style=custompl, firstline=27, lastline=27]{Codigos/codigosProlog.pl}
Cuando maxDepth alcanza el valor 1 comprobamos que la condici\'on se cumple y llamamos a ejecutar el resto de las instrucciones.

\item \textbf{Caso 2:} caso en el que la condici\'on no se hace cierta, es decir hemos llegado al fin del bucle por lo que llamamos a ejecutar al resto de las instrucciones.
\lstinputlisting[caption=Instrucci\'on para el caso en el que llegamos al fin del bucle, style=custompl, firstline=29, lastline=29]{Codigos/codigosProlog.pl}

\item \textbf{Caso 3:} caso en el que la condici\'on se hace cierta y a\'un no hemos llegado a la profundidad m\'axima permitida por lo que ejecutamos el cuerpo del bucle.
\lstinputlisting[caption=Instrucci\'on para el caso en el que ejecutamos el cuerpo del bucle, style=custompl, firstline=31, lastline=31]{Codigos/codigosProlog.pl}
\end{itemize}

%-------------------------------------------------------------------
\section{Instrucciones de Entrada/Salida.}
%-------------------------------------------------------------------

Como ya dijimos anteriormente, iremos almacenando en cada paso el estado actual que est\'a  formado por: Table (la tabla de variables), Cin (Entrada por consola), Cout (Salida por consola), Trace (Traza o camino que llevamos ejecutado).

%-------------------------------------------------------------------
\subsection*{ConsoleIn (Entrada)}
%-------------------------------------------------------------------

Para recoger una expresi\'on sencilla que viene como entrada por consola lo haremos mediante el predicado:

\lstinputlisting[caption=Instrucci\'on para la ejecuci\'on de la consola de entrada, style=custompl, firstline=47, lastline=47]{Codigos/codigosProlog.pl}

Recogeremos el valor contenido en la variable Value y lo concatenaremos al resto de la entrada que llevemos almacenado en Cin.

%-------------------------------------------------------------------
\subsection*{ConsoleOut (Salida)}
%-------------------------------------------------------------------

Para devolver un valor por consola lo haremos mediante el predicado:

\lstinputlisting[caption=Instrucci\'on para la ejecuci\'on de la consola de salida, style=custompl, firstline=17, lastline=17]{Codigos/codigosProlog.pl}

Nos sirve para sacar por pantalla el resultado de la expresi\'on, por lo que llamaremos a la funci\'on {\it resuelveExpresi\'on} con la cxpresi\'on como par\'ametro y el resultado devuelto ser\'a incluido en el Cout anteriormente mencionado de forma que al final saldr\'a por pantalla todo aquello que se encuentre en esa variable. Ejecutamos el resto de las instrucciones.

%-------------------------------------------------------------------
\section{Instrucci\'on de retorno.}
%-------------------------------------------------------------------

%-------------------------------------------------------------------
\subsection*{Instrucci\'on RETURN}
%-------------------------------------------------------------------

\'esta es la sentencia de salida de una funci\'on, cuando se ejecuta, se devuelve el control a la rutina que llam\'o a la funci\'on. Adem\'as, se usa para especificar el valor de retorno de la funci\'on. La instrucci\'on de return sigue el siguiente esquema:

\begin{center}
\textbf{return [<expresi\'on>] ;}
\end{center}

La instrucci\'on de retorno del valor de una funci\'on se recoge en el predicado:

\lstinputlisting[caption=Instrucci\'on para la ejecuci\'on de la sentencia return, style=custompl, firstline=19, lastline=19]{Codigos/codigosProlog.pl}

En la tabla de variables llevamos reservada una variable que nos servir\'a como variable de retorno por lo tanto en este predicado lo \'unico que haremos ser\'a ejecutar con la llamada a execute del cuerpo del return y con el valor devuelto se actualizar\'a esa variable reservada. En este caso no llamaremos a ejecutar al resto de instrucciones ya que, como se trata de una instrucci\'on de retorno, es la \'ultima instrucci\'on que se deber\'ia de contemplar en esta funci\'on por tanto no deber\'iamos de seguir con el flujo del programa.\\\\\\


%-------------------------------------------------------------------
\section{Uso del Int\'erprete}
%-------------------------------------------------------------------

El int\'erprete se puede ejecutar mediante el siguiente comando:

\begin{center}
interpreter(EntryFile, OutFile, Inf, Sup, MaxDepth, FunctionName).
\end{center}

Donde se puede simular con valores espec\'ificos en las variables Inf, Sup y MaxDepth.

\begin{center}
interpreter(EntryFile, OutFile, FunctionName).
\end{center}

Donde Inf toma el valor -5, Sup toma el valor 5 y MaxDepth toma el valor 10. [30]

% Variable local para emacs, para  que encuentre el fichero maestro de
% compilaci\'on y funcionen mejor algunas teclas r\'apidas de AucTeX
%%%
%%% Local Variables:
%%% mode: latex
%%% TeX-master: "../Tesis.tex"
%%% End:
