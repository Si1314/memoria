%---------------------------------------------------------------------
%
%                          Cap�tulo 1
%
%---------------------------------------------------------------------

\chapter{Conclusiones y trabajo futuro}

\pagestyle{plain}

%-------------------------------------------------------------------
\section{Conclusiones}
%-------------------------------------------------------------------

%-------------------------------------------------------------------
\section*{Conclusiones sobre el trabajo realizado}
%-------------------------------------------------------------------

Tras un curso desarrollando este proyecto podemos decir que hemos cumplido los objetivos fijados al inicio del curso, desarrollando una aplicaci�n que realiza la ejecuci�n simb�lica de un conjunto limitado de operadores de C++. La herramienta eval�a todas las posibles ramas de funciones sencillas teniendo en cuenta todas las limitaciones de la ejecuci�n.

Los resultados obtenidos por dicha ejecuci�n se muestran de una forma sencilla y visual al usuario, adem�s de mostrar las l�neas que recorre dicho int�rprete.


%-------------------------------------------------------------------
\section*{Conclusiones personales}
%-------------------------------------------------------------------

Este proyecto nos ha aportado ciertas capacidades que no hemos podido desarrollar en otras asignaturas. En primer lugar, nos ha permitido enfrentarnos a un proyecto de investigaci�n inform�tica, mucho m�s te�rico que pr�ctico. Donde el mayor trabajo se ha centrado en la investigaci�n de tecnolog�as que en el desarrollo de las mismas. 

Por otro lado, el hecho de enfrentarnos nosotros mismos a un proyecto, a pesar de la tutorizaci�n, nos ha obligado a trabajar en equipo, coordinarnos y organizarnos para cumplir unos objetivos y plazos estipulados por nosotros mismos. Eso nos ha aportado responsabilidad y compromiso para con el proyecto.

A pesar que durante el presente curso hemos cursado otras asignaturas y no hemos tenido la posibilidad de dedicarle m�s tiempo al proyecto, consideramos que el trabajo realizado ha sido muy productivo y nos ha aportado una gran experiencia acad�mica y personal.


%-------------------------------------------------------------------
\section{Ampliaciones potenciales y trabajo futuro}
%-------------------------------------------------------------------

Nuestro proyecto puede ser utilizado como base para siguientes cursos para la asignatura de Sistemas Inform�ticos o proyecto de final de grado. Desde el primer d�a, nuestro tutor nos puntualiz� que este es trabajo para toda una vida, y tener en cuenta toda la sintaxis de C++ es inviable en un curso acad�mico.

%-------------------------------------------------------------------
\section*{Mayor repertorio de instrucciones}
%-------------------------------------------------------------------

Como hemos comentado, en este proyecto hemos trabajado con un peque�o repertorio de las instrucciones de C++, puesto que abarcar toda su sintaxis era inviable. Es por esto que trabajamos con instrucciones b�sicas y el tipo de n�meros enteros.

Si esta aplicaci�n se quisiese usar como un depurador completo, ser�a necesario contemplar toda la sintaxis de C++ y todos los tipos de variables.

%-------------------------------------------------------------------
\section*{Otros sistemas operativos}
%-------------------------------------------------------------------

Una de las limitaciones que tuvimos a la hora de desarrollar la aplicaci�n es la instalaci�n del compilador Clang en el sistema operativo de Windows, por tanto nos vimos obligados a limitarnos a desarrollar en Linux. Esta incompatibilidad limita la portabilidad de la herramienta y por tanto el n�mero de usuarios que podr�an utilizarla.

Si la apliacaci�n quisiese abarcar a una mayor comunidad de usuarios ser�a necesario investigar el modo de compilar la herramienta de Clang en otros sistemas operativos como Windows o MacOSX.

%-------------------------------------------------------------------
\section*{Aplicacion web}
%-------------------------------------------------------------------

En referencia a la secci�n anterior, una posible ampliaci�n ser�a exportar esta aplicaci�n al �mbito web. De este modo, no ser�a necesario instalaci�n de librer�as u otros programas que tengan incompatibilidades con el sistema operativo, y por otro lado los usuarios no s�lo se limitar�an a los que conociesen la aplicaci�n por la Universidad sino que la comunidad potencial es global.



% Variable local para emacs, para  que encuentre el fichero maestro de
% compilaci�n y funcionen mejor algunas teclas r�pidas de AucTeX
%%%
%%% Local Variables:
%%% mode: latex
%%% TeX-master: "../Tesis.tex"
%%% End:
