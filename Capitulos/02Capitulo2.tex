%---------------------------------------------------------------------
%
%                          Cap\'itulo 2
%
%---------------------------------------------------------------------

\chapter{Herramienta de Clang}

\pagestyle{plain}

En este cap\'itulo analizaremos en profundidad la herramienta construida para realizar el primer paso hacia una ejecuci\'on simb\'olica de un programa por nuestro proyecto. Discutiremos las decisiones de dise\~no y los m\'etodos empleados para organizar los datos de una manera intuitiva y c\'omoda de manipular en la siguiente etapa del proceso. 

La herramienta AST2XML conforma junto con el int\'erprete simb\'olico el cuerpo central del proyecto. Si bien esta herramienta no cumple el papel m\'as importante, suple las necesidades b\'asicas de obtener a partir del c\'odigo fuente una representaci\'on m\'as manejable y sencilla de visualizar. Ha sido desarrollada a partir del proyecto abierto e internacional Low Level Virtual Machine (a partir de ahora: LLVM), pero se apoya principalmente en su compilador CLANG.

Al inicio del curso observamos que nos pod\'iamos aproximar al problema de distintas formas. Si bien, de una forma similar al proyecto jSyx teniamos la posibilidad de basar nuestro proyecto en el uso de algo similar al bytecode de java pero relacionado con C++. De hecho tal aproximaci\'on hacia el an\'alisis simb\'olico existe en la m\'aquina virtual simb\'olica Klee, el cual deriva del proyecto LLVM. 

Independientemente de ello, optamos por una aproximaci\'on diferente, que es separar expresamente el int\'erprete de la herramienta que analiza el c\'odigo. Por tanto viendo las posibilidades que aporta el proyecto LLVM decidimos enfocarlo de manera similar.


%-------------------------------------------------------------------
\section{Historia de LLVM y CLANG}
%-------------------------------------------------------------------

%-------------------------------------------------------------------
\subsection{LLVM}
%-------------------------------------------------------------------

El un principio ``LLVM'' eran las iniciales de ``Low Level Virtual Machine'' (M\'aquina Virtual de Bajo Nivel). LLVM naci\'o como proyecto en el a\~no 2000 en la Universidad de Illinois en UrbanaChampaign. En 2005, junto con Apple Inc., se empez\'o a adaptar el sistema para varios usos dentro del ecosistema de desarrollo de Apple. La denominaci\'on inicial del proyecto con las siglas LLVM caus\'o una confusi\'on ampliamente difundida, puesto que las m\'aquinas virtuales son s\'olo una de sus aplicaciones posibles. Cuando la extensi\'on del proyecto se ampli\'o incluso m\'as, LLVM se convirti\'o en un proyecto que engloba una gran varriedad de otros compiladores y tecnolog\'ias de bajo nivel. Por tanto el proyecto abandon\'o las iniciales y actualmente, LLVM es la manera de referirse a todas esas utilidades. Actualmente LLVM est\'a integrado en las \'ultimas herramientas de desarrollo de Apple para sus sistemas operativos.

%-------------------------------------------------------------------
\subsection{CLANG}
%-------------------------------------------------------------------

El amplio inter\'es que ha recibido LLVM ha llevado a una serie de tentativas para desarrollar frontends totalmente nuevos para una variedad de lenguajes. El que ha recibido la mayor atenci\'on es CLANG, un nuevo compilador que soporta otros lenguajes de la familia de C (ObjectiveC, C++, etc ...).

En un principio estaba proyectado que LLVM se apoyara en el compilador de C de GNU [19] (a partir de ahora GCC) pero los desarrolladores de Apple fueron encontrando numerosas dificultades t\'ecnicas. Principalmente el problema fue que al ser una plataforma desarrollada durante mucho tiempo GCC presentaba una estructura demasiado compleja sobre la que ir a\~nadiendo las modificaciones necesarias para solventar los problemas de compatibilidad.

CLANG est\'a dise\~nado para trabajar en una capa de abstracci\'on por encima de LLVM. Una de los principales objetivos del dise\~no de CLANG es dar soporte a la compilaci\'on incremental1 con la idea de conseguir una completa integraci\'on de esta a los entornos de desarrollo con interfaz gr\'afica.

CLANG est\'a construido siguiendo una estructura modular basada en la inclusi\'on de numerosas bibliotecas con lo que intenta alejarse de otros compiladores de dise\~no monol\'itico. Este dise\~no modular se adapta a la escalabilidad que se buscaba originalmente en el proyecto. CLANG tambi\'en fue dise\~nado para conservar mucha m\'as informaci\'on durante la compilaci\'on que GCC y mantener la forma original del c\'odigo. Asimismo se ha dado importancia a los informes de error que son bastante detallados.

Actualmente es apoyado principalmente por Apple y se espera que CLANG sustituya al comiplador del sistema GCC.

%-------------------------------------------------------------------
\section{Uso de Clang}
%-------------------------------------------------------------------

En la mayor\'ia de los casos CLANG ejecutar\'a el preprocesado y parsear\'a el c\'odigo formando un \'arbol de sint\'axis abstracta (de ahora en adelante: AST). Esta estructura de \'arbol es m\'as manejable que el propio c\'odigo con el a\~nadido de que cualquier subestructura mantiene referencias al c\'odigo fuente.

En nuestro proyecto el uso de CLANG ha sido bastante localizado dentro de las funcionalidades que ofrece.

%-------------------------------------------------------------------
\subsection{AST}
%-------------------------------------------------------------------

Un AST es una representaci\'on de \'arbol de la estructura sint\'actica abstracta simplificada del c\'odigo fuente. Esta estructura es ampliamente utilizada en los distintos compiladores del mercado. Cada nodo del \'arbol denota una construcci\'on de lo que se encuenta en el c\'odigo fuente.

La herramienta AST2XML se apoya en la adaptaci\'on del c\'odigo, por parte del AST propio de CLANG, a un formato XML. La representaci\'on del AST que emplea CLANG se diferencia de la de otros compiladores en cuanto a que mantiene la semejanza con el c\'odigo escrito y con el est\'andar de C++.

\figura{Vectorial/figura1}{width=.9\textwidth}{fig:figura1}%
{Figura Ejemplo del AST que emplea Clang a partir de un c\'odigo espec\'ifico.}

Como se puede apreciar en la figura, la representaci\'on que se obtiene directamente de CLANG puede llegar a parecer abrumadoramente compleja y redundante. En momentos iniciales del proyecto tomamos la decisi\'on de que informaci\'on como la ubicaci\'on de los elementos, los tipos de las variables, o incluso el tipo de los nodos no ser\'ian necesarias en cada momento. Para ello hizo falta una implementaci\'on muy precisa de un tipo de herramienta que se puede construir para CLANG.

%-------------------------------------------------------------------
\subsection{La biblioteca LibTooling}
%-------------------------------------------------------------------

La biblioteca LibTooling tiene un papel muy importante en la arquitectura de la herramienta. Libtooling da soporte al dise\~no de herramientas aut\'onomas basadas en CLANG, proveyendo de la infraestructura necesaria para la realizaci\'on de an\'alisis sint\'acticos y sem\'anticos de programas. [20]

En la implementaci\'on de nuestra herramienta destacan los roles de estos elementos que nos aporta la biblioteca:

\begin{itemize}
\item \textbf{CommonOptionParser:} Es una utilidad que se encarga del an\'alisis de
los argumentos que recibe la llamada del ejecutable ast2xml. Aporta los
mensajes por defecto de las utilidades b\'asicas de las herramientas que
se pueden implementar.
\item \textbf{ToXMLVisitor:} Es el n\'ucleo de la herramienta. Hereda de la interfaz RecursiveASTVisitor [21], que como su nombre indica, se encarga de hacer un recorrido recursivo y en profundidad de la estructura. Dentro de nuestro visitor hay que implementar los diferentes m\'etodos espec\'ificos que se amoldan a los dist\'intos nodos que vaya a visitar dentro del AST.
\item \textbf{ToXMLASTConsumer:} Es la clase que desencadena la ejecuci\'on del visitor.
\item \textbf{ToXMLFrontendAction:} Es la clase que encapsula el comportamiento de nuestra herramienta. Hereda de la clase abstracta ASTFrontendAction del paquete FrontendAction. Devuelve una instancia de nuestro consumidor en el momento de comenzar a recorrer el AST.
\end{itemize}

%-------------------------------------------------------------------
\section{Estructura y recorrido del AST}
%-------------------------------------------------------------------

La estructura que emplea CLANG para representar el AST de un c\'odigo se basa principalmente en la interacci\'on de dos clases base muy flexibles a partir de las cuales se construyen todas los dem\'as: Decl (Declarations), que engloba toda las declaraciones (funciones, variables, templates, etc ...), Stmt (Statement) que abarca las instrucciones. La mayor\'ia de las clases que se derivan de ellas se explican por s\'i mismas, como por ejemplo: BinaryOperator (operador binario), FunctionDecl (declaraci\'on de funci\'on), etc ... [22]

En el desarrollo de la herramienta implementada empleamos un objeto de la clase propia ToXMLVisitor que hereda de la clase RecursiveASTVisitor cuyo comportamiento est\'a previamente definido dentro de las bibliotecas propias de CLANG.

El recorrido que hace el RecursiveASTVisitor a lo largo del AST se realiza mediante llamadas a funciones Traverse. Estas funciones son espec\'ificas para cada clase que funciona como nodo del AST, aunque existen sus versiones gen\'ericas: TraverseDecl() y TraverseStmt().

El comportamiento de ambas funciones es muy similar. En el caso de la funci\'on TraverseStmt, la instancia del RecursiceASTVisitor realiza llamadas recursivas a los nodos del AST mediante las funciones TraverseXXXStmt. Siendo XXX el nombre de la clase del nodo en cuesti\'on. En cada llamada el RecursiceASTVisitor hace una comprobaci\'on del tipo del nodo y de si la funci\'on necesaria est\'a declarada. En caso de no estar declarada expl\'icitamente en el c\'odigo se prosigue con la b\'usqueda accediendo a los nodos m\'as profundos.

Para realizar un recorrido satisfactorio es necesario implementar en el c\'odigo de la herramienta las distintas funciones Traverse de los nodos de inter\'es. \'esto es posible ya que las clases de las cuales los nodos son instancias particulares heredan de las super clases Decl y Stmt. Con nuestra herramienta transformamos cada nodo del AST en un nodo XML, manteniendo la sem\'antica y la estructura del c\'odigo.

%-------------------------------------------------------------------
\subsection{Funciones}
%-------------------------------------------------------------------

En los ficheros de c\'odigo que se espera que introduzcan los usuarios la estructura m\'as grande a tener en cuenta es la declaraci\'on de funciones. El compilador se comporta de igual manera tanto si se trata de la declaraci\'on de una funci\'on a la que se va a invocar, como si es el caso del main del c\'odigo.

%-------------------------------------------------------------------
\subsubsection*{FunctionDecl}
%-------------------------------------------------------------------

La clase FunctionDecl se corresponde con la declaraci\'on de una funci\'on. Es un nodo del AST de CLANG que almacena informaci\'on correspondiente a la situaci\'on en el c\'odigo de \'esta, el tipo devuelto, enlaces a sus par\'ametros y al cuerpo de la funci\'on. Este nodo se corresponde directamente en el XML devuelto por la herramienta con los nodos function.

%-------------------------------------------------------------------
\subsubsection*{ParmVarDecl}
%-------------------------------------------------------------------

Un objeto ParmVarDecl almacena informaci\'on referente a un par\'ametro necesario en la llamada de una funci\'on. Esta clase se corresponde con el nodo XML param, el cual contiene el tipo y el nombre del par\'ametro dentro de la funci\'on. Los nodos param s\'olo aparecen dentro de un nodo params.

%-------------------------------------------------------------------
\subsubsection*{CompoundStmt}
%-------------------------------------------------------------------

Una noci\'on que existe en el AST de CLANG es la del CompoundStatement que representa a un bloque de c\'odigo delimitado por llaves ({...}). Puesto que esperamos que el usuario haga uso de ellos s\'olo en compa\~nia de otras estructuras (if, while, for o funciones), en el XML se representa con el nodo body. Un nodo body est\'a compuesto por la serie de nodos correspondientes a las instrucciones que lo conforman.

La disposici\'on b\'asica de un nodo function se puede observar en los siguientes listados de c\'odigo, 2.1 con el c\'odigo C++ origen y 2.2 con el c\'odigo XML con la disposici\'on b\'asica de un nodo function.

\lstinputlisting[caption=C\'odigo C++ ejemplo para un nodo function, style=customxml, firstline=1, lastline=3]{Codigos/ccCap2.cpp}
\lstinputlisting[caption=C\'odigo XML obtenido\, ejemplo de la disposici\'on b\'asica de un nodo function, style=customxml, firstline=1, lastline=10]{Codigos/xmlsCap2.xml}

%-------------------------------------------------------------------
\subsubsection*{ReturnStmt}
%-------------------------------------------------------------------

La instrucci\'on Return se traduce en nuestro formato a un nodod return el cual contiene su l\'inea del c\'odigo y un nodo interno que contiene la expresi\'on que devuelve la funci\'on.

La disposici\'on b\'asica de un nodo return se puede observar en los siguientes listados de c\'odigo, 2.3 con el c\'odigo C++ origen y 2.4 con el c\'odigo XML con la disposici\'on b\'asica de un nodo return.

\lstinputlisting[caption=C\'odigo C++ ejemplo para un nodo return, style=customxml, firstline=5, lastline=5]{Codigos/ccCap2.cpp}
\lstinputlisting[caption=C\'odigo XML obtenido\, ejemplo de la disposici\'on b\'asica de un nodo return, style=customxml, firstline=12, lastline=16]{Codigos/xmlsCap2.xml}

%-------------------------------------------------------------------
\subsection{Declaraciones e inicializaciones}
%-------------------------------------------------------------------

C++ es un lenguaje fuertemente tipado, y requiere que cada variable est\'e declarada junto con su tipo antes de su primer uso. En un caso pr\'actico, esto informa al compilador el tama\~no reservar en memoria para la variable y la forma de interpretar su valor. Si se declaran varias variables del mismo tipo, se puede realizar en una sola instrucci\'on.

%-------------------------------------------------------------------
\subsubsection*{DeclStmt}
%-------------------------------------------------------------------

Cuando se declara una variable, adquiere un valor indeterminado hasta que se le asigne alguno expl\'icitamente. En C++ se contemplan tres maneras de inicializar el valor de una variable, aunque en nuestro proyecto, de cara a el uso de nuestra herramienta por parte de alumnos que est\'an aprendiendo a programar, admitimos \'unicamente la forma m\'as sencilla.

Un nodo declarations contiene las declaraciones de las variables de un mismo tipo que han sido declaradas en la misma l\'inea. Obtienen su informaci\'on del nodo del AST de CLANG DeclStmt.

%-------------------------------------------------------------------
\subsubsection*{VarDecl}
%-------------------------------------------------------------------

Los nodos declaration contienen informaci\'on acerca del tipo de la variable, su nombre, la l\'inea en la que se encuentra la declaraci\'on y la expresi\'on cuyo valor se le va a asignar. La informaci\'on surge de los nodos VarDecl que emplea el AST de CLANG.

La disposici\'on b\'asica de un nodo declaration se puede observar en los siguientes listados de c\'odigo, 2.5 con el c\'odigo C++ origen y 2.6 con el c\'odigo XML con la disposici\'on b\'asica de un nodo declaration.

\lstinputlisting[caption=C\'odigo C++ ejemplo para un nodo declaration, style=customxml, firstline=7, lastline=8]{Codigos/ccCap2.cpp}
\lstinputlisting[caption=C\'odigo XML obtenido\, ejemplo de la disposici\'on b\'asica de un nodo declaration, style=customxml, firstline=18, lastline=27]{Codigos/xmlsCap2.xml}

%-------------------------------------------------------------------
\subsection{Estructuras de control}
%-------------------------------------------------------------------

Las instrucciones que producen ramificaciones en la ejecuci\'on de un programa, tales como: el if, el while o el for. Para poder realizar su ejecuci\'on correctamente mantenemos en sus respectivos nodos sus se\~nas m\'as caracter\'isticas.

%-------------------------------------------------------------------
\subsubsection*{IfStmt}
%-------------------------------------------------------------------

El nodo XML if obtiene su informaci\'on a partir del nodo del AST de CLANG IfStmt. Almacena el n\'umero de l\'inea, y los nodos correspondientes a la condici\'on que dirige su ejecuci\'on, al cuerpo del then y al cuerpo del else (en caso de estar especificado en el c\'odigo).

La disposici\'on b\'asica de un nodo If se puede observar en los siguientes listados de c\'odigo, 2.7 con el c\'odigo C++ origen y 2.8 con el c\'odigo XML con la disposici\'on b\'asica de un nodo If.

\lstinputlisting[caption=C\'odigo C++ ejemplo para un nodo If, style=customxml, firstline=10, lastline=14]{Codigos/ccCap2.cpp}
\lstinputlisting[caption=C\'odigo XML obtenido\, ejemplo de la disposici\'on b\'asica de un nodo If, style=customxml, firstline=29, lastline=44]{Codigos/xmlsCap2.xml}

%-------------------------------------------------------------------
\subsubsection*{WhileStmt}
%-------------------------------------------------------------------

La instrucci\'on while se traduce a un nodo que alberga como atributo el n\'umero de l\'inea donde comienza la instrucci\'on, y los nodos de la expresi\'on de su condici\'on y el bloque de instrucciones de su cuerpo. Toda esta informaci\'on es una fracci\'on de la que contiene el nodo del AST de clang WhileStmt.

La disposici\'on b\'asica de un nodo While se puede observar en los siguientes listados de c\'odigo, 2.9 con el c\'odigo C++ origen y 2.10 con el c\'odigo XML con la disposici\'on b\'asica de un nodo While.

\lstinputlisting[caption=C\'odigo C++ ejemplo para un nodo While, style=customxml, firstline=16, lastline=18]{Codigos/ccCap2.cpp}
\lstinputlisting[caption=C\'odigo XML obtenido\, ejemplo de la disposici\'on b\'asica de un nodo While, style=customxml, firstline=46, lastline=54]{Codigos/xmlsCap2.xml}

%-------------------------------------------------------------------
\subsubsection*{ForStmt}
%-------------------------------------------------------------------

El bucle For se representa en el fichero XML con un nodo for. Este nodo obtiene su informaci\'on a partir del nodo del AST de clang ForStmt. La informaci\'on que contiene refiere al n\'umero de l\'inea en el que comienza, y a los nodos de la declaraci\'on de su variable de control, la condici\'on de parada, la instrucci\'on de avance y el cuerpo del bucle. Es importante recalcar que para el uso correcto de la herramienta, m\'as espec\'ificamente para el uso del int\'erprete simb\'olico, el bucle for introducido por el usuario en el c\'odigo fuente debe contener dichas partes de forma expl\'icita.

La disposici\'on b\'asica de un nodo For se puede observar en los siguientes listados de c\'odigo, 2.11 con el c\'odigo C++ origen y 2.12 con el c\'odigo XML con la disposici\'on b\'asica de un nodo For.

\lstinputlisting[caption=C\'odigo C++ ejemplo para un nodo For, style=customxml, firstline=20, lastline=22]{Codigos/ccCap2.cpp}
\lstinputlisting[caption=C\'odigo XML obtenido\, ejemplo de la disposici\'on b\'asica de un nodo For, style=customxml, firstline=56, lastline=75]{Codigos/xmlsCap2.xml}

%-------------------------------------------------------------------
\subsection{Expresiones}
%-------------------------------------------------------------------

Las expresiones escritas en C++ se traducen a nuestro formato en XML en nodos descritos por sus operadores, la variable a la que hacen referencia o la constante que representan.

En caso de que estos nodos hagan referencia a un operador, contienen informaci\'on acerca del tipo de este (binario, unario, asignaci\'on, etc ...) y de las expresiones que contienen.

Las expresiones que se refieren a variables almacenan el nombre de estas y deforma equivalente las que representan constantes guardan su valor. Un caso particular de expresi\'on ser\'ia el de las llamadas a funciones.

%-------------------------------------------------------------------
\subsubsection*{ParenExpr}
%-------------------------------------------------------------------

La clase ParenExpr representa una expresi\'on rodeada por par\'entesis por ejemplo ``(1)''. Esta estructura no es muy relevante en nuestra herramienta porque en el paso del an\'alisis en el que CLANG construye el AST, ya tiene en cuenta la prioridad de los operandos. Esto convierte a la clase ParenExpr en una utilidad auxiliar que sin duda puede resultar \'util en caso de querer modificar o a\~nadir nuevos elementos al AST en otro tipo de an\'alisis con otra herramienta.

%-------------------------------------------------------------------
\subsubsection*{BinaryOperator}
%-------------------------------------------------------------------

La clase BinaryOperator engloba los diversos operadores binarios de C++. Abarca las operaciones con punteros, la asignaci\'on, distintas operaciones num\'ericas, booleanas, de comparaci\'on e incluso a nivel de bit. CLANG se refiere a los distintos operadores mediante un tipo enumerado denominado OpCode, pero en nuestra herramienta optamos por hacer una distinci\'on entre la asignaci\'on y los dem\'as operadores. De esta forma una asignaci\'on se ver\'ia traducidad a un nodo XML assignment que contiene el nombre de la variable que recibe el valor, su ubicaci\'on en el c\'odigo y un nodo interno para la expresi\'on de dicho valor. De esta forma no se contempla que haya una asignaci\'on dentro de otra. Asimismo cualquier otra expresi\'on aparece como un nodo binaryOperator que aparte de conservar el tipo y el operador, contiene los dos nodos internos que refieren a las expresiones que hacen a su vez el papel de sus dos operandos. Esta distinci\'on se hace con los estudiantes noveles de programaci\'on en mente con la finalidad de que, en caso de estudiar detenidamente el XML final de su c\'odigo, tengan una visi\'on m\'as clara de su funcionamiento.

La disposici\'on b\'asica de un nodo con operador binario se puede observar en los siguientes listados de c\'odigo, 2.13 con el c\'odigo C++ origen y 2.14 con el c\'odigo XML con la disposici\'on b\'asica de un nodo para un operador binario.

\lstinputlisting[caption=C\'odigo C++ ejemplo para un nodo de un operador binario, style=customxml, firstline=24, lastline=24]{Codigos/ccCap2.cpp}
\lstinputlisting[caption=C\'odigo XML obtenido\, ejemplo de la disposici\'on b\'asica de un nodo con operador binario, style=customxml, firstline=77, lastline=82]{Codigos/xmlsCap2.xml}

%-------------------------------------------------------------------
\subsubsection*{CompoundAssignOperator}
%-------------------------------------------------------------------

CLANG hace distinci\'on entre los operadores binarios corrientes, y los que a su vez almacenan el valor resultante de la operaci\'on en una de las variables involucradas. Operaciones de este este tipo se resuelven mediante procesos de bajo nivel distintos a los de BinaryOperator por lo que para ello la clase CompoundAssignOperator resulta bastante \'util. Un nodo CompoundAssignOperator se traduce a un nodo assignmentoperator en el fichero XML devuelto. En este nodo damos cuenta del nombre de la variable de la que participa en la operaci\'on y posteriormente recoge el valor, el tipo de la operaci\'on, la ubicaci\'on en el c\'odigo y un nodo anidado que corresponder\'ia a la expresi\'on que pudiera aparecer en el c\'odigo correspondiendo al segundo operando.

La disposici\'on b\'asica de un nodo con operador de asignaci\'on se puede observar en los siguientes listados de c\'odigo, 2.15 con el c\'odigo C++ origen y 2.16 con el c\'odigo XML con la disposici\'on b\'asica de un nodo para un operador de asignaci\'on.

\lstinputlisting[caption=C\'odigo C++ ejemplo para un nodo de un operador de asignaci\'on, style=customxml, firstline=26, lastline=26]{Codigos/ccCap2.cpp}
\lstinputlisting[caption=C\'odigo XML obtenido\, ejemplo de la disposici\'on b\'asica de un nodo con operador de asignaci\'on, style=customxml, firstline=84, lastline=86]{Codigos/xmlsCap2.xml}

%-------------------------------------------------------------------
\subsubsection*{UnaryOperator}
%-------------------------------------------------------------------

En C++ existen numerosos operadores unarios pero, considerando los conocimientos que hacen falta en etapas tempranas de la iniciaci\'on a la programaci\'on, nos hemos centrado en dar soporte a los operadores unarios de incremento, decremento, indicador de signo y el not l\'ogico. Distinguimos los operadores de incremento y decremento se presentan a los programadores en un principio como parte esencial de los bucles cumpliendo la funci\'on de instrucci\'on de avance. Estos operadores se traducen a un nodo unaryOperator que indica el tipo de la operaci\'on: ``+'' en caso de incremento, ``'' en caso contrario.

Por otro lado el indicador de signo cumple un papel importante a la hora de operar con valores negativos y de ayuda expl\'icita al programador junior en caso de que quiera asegurar el valor de su variable. El nodo XML correspondiente es el nodo signOperator, que indica si es positivo o negativo y que contiene un subnodo correspondiente a la expresi\'on cuyo signo se modifica.

El not l\'ogico no aporta m\'as expresividad pero permite simplificar programaci\'on de expresiones booleanas. Este operador aparece en el formato XML como un nodo signOperator. Este nodo contiene el nodo de la expresi\'on cuyo valor booleano invierte.

%-------------------------------------------------------------------
\subsubsection*{ImplicitCastExpr}
%-------------------------------------------------------------------

Esta clase nos permite representar expl\'icitamente las conversiones de tipo (castings) que no tienen representaci\'on en el c\'odigo original. Un ejemplo en la pr\'actica ser\'ia a la hora de determinar el tipo de una variable de la cual solo sabemos el indentificador de forma que case con el tipo de la operaci\'on a la que va a ser sometida. En el lenguaje C, los conversiones impl\'icitas de tipo siempre producen valores temporales que van asociados al lado derecho de una expresi\'on (rvalues), por ejemplo: ``x = y'', la variable y es un rvalue. A\'un as\'i, en C++ una conversi\'on impl\'icita puede estar asociada a valores que pueden ir a la izquierda de una expresi\'on (lvalues), por ejemplo: ``x = 7'', x es un lvalue.

%-------------------------------------------------------------------
\subsubsection*{DeclRefExprClass}
%-------------------------------------------------------------------

La clase DeclRefExprClass se corresponde con una referencia a una variable, funci\'on, valor de tipo enumerado, etc. que haya sido declarado con anterioridad. Un objeto de esta clase almacena numerosos detalles de como una declaraci\'on se referencia dentro de una expresi\'on. En nuestro proyecto este nodo nos permite conocer el identificador de una variable cualquiera que participe en una expresi\'on. La correspondencia con el fichero XML tiene forma de nodo variable.

La disposici\'on b\'asica de un nodo con una asignaci\'on a otra variable se puede observar en los siguientes listados de c\'odigo, 2.17 con el c\'odigo C++ origen y 2.18 con el c\'odigo XML con la disposici\'on b\'asica de un nodo con una asignaci\'on a otra variable.

\lstinputlisting[caption=C\'odigo C++ ejemplo para un nodo con una asignaci\'on a otra variable, style=customxml, firstline=28, lastline=28]{Codigos/ccCap2.cpp}
\lstinputlisting[caption=C\'odigo XML obtenido\, ejemplo de la disposici\'on b\'asica de un nodo con una asignaci\'on a otra variable, style=customxml, firstline=88, lastline=90]{Codigos/xmlsCap2.xml}

%-------------------------------------------------------------------
\subsubsection*{IntegerLiteral}
%-------------------------------------------------------------------

Un nodo IntegerLiteral es utilizado por el compilador para determinar el valor y el tipo asociado al tama\~no que ocupa en memoria un valor entero que aparece expl\'icitamente en el c\'odigo. En el formato XML correspondiente al c\'odigo fuente que acepta nuestra herramienta se traduce en un nodo const en el que se almacena el valor de dicha constante.

La disposici\'on b\'asica de un nodo con una asignaci\'on a un valor se puede observar en los siguientes listados de c\'odigo, 2.19 con el c\'odigo C++ origen y 2.20 con el c\'odigo XML con la disposici\'on b\'asica de un nodo con una asignaci\'on a un valor.

\lstinputlisting[caption=C\'odigo C++ ejemplo para un nodo con una asignaci\'on a un valor, style=customxml, firstline=30, lastline=30]{Codigos/ccCap2.cpp}
\lstinputlisting[caption=C\'odigo XML obtenido\, ejemplo de la disposici\'on b\'asica de un nodo con una asignaci\'on a un valor, style=customxml, firstline=92, lastline=94]{Codigos/xmlsCap2.xml}

%-------------------------------------------------------------------
\subsubsection*{StringLiteral}
%-------------------------------------------------------------------

La clase StringLiteral sirve para representar cualquier expresion de tipo string (cadenas de caracteres). Es una clase muy vers\'atil pues otorga al programador distintas utilidades para manipular strings a distintos niveles. En nuestra herramienta la presencia de strings definidos por el usuario est\'a limitada a la salida por consola. Un string se traduce a un nodo string cuyo valor almacena la cadena de caracteres correspondiente

La disposici\'on b\'asica de un nodo con un valor string se puede observar en los siguientes listados de c\'odigo, 2.21 con el c\'odigo C++ origen y 2.22 con el c\'odigo XML con la disposici\'on b\'asica de un nodo con un valor string.

\lstinputlisting[caption=C\'odigo C++ ejemplo para un nodo con uun valor string, style=customxml, firstline=32, lastline=32]{Codigos/ccCap2.cpp}
\lstinputlisting[caption=C\'odigo XML obtenido\, ejemplo de la disposici\'on b\'asica de un nodo con un valor string, style=customxml, firstline=96, lastline=96]{Codigos/xmlsCap2.xml}

%-------------------------------------------------------------------
\subsubsection*{CallExpr}
%-------------------------------------------------------------------

La clase CallExpr representa la llamada a una funci\'on. En C++ las llamadas a funciones son consideradas expresiones que devuelven un valor del tipo asociado a dicha funci\'on. Una llamada a funci\'on se traduce a un nodo callFunction que refiere al nombre de la funci\'on y contiene varios nodos arg con las expresiones que el programador haya determinado.

La disposici\'on b\'asica de un nodo con con la llamada a una funci\'on se puede observar en los siguientes listados de c\'odigo, 2.23 con el c\'odigo C++ origen y 2.24 con el c\'odigo XML con la disposici\'on b\'asica de un nodo con con la llamada a una funci\'on.

\lstinputlisting[caption=C\'odigo C++ ejemplo para un nodo con la llamada a una funci\'on, style=customxml, firstline=34, lastline=34]{Codigos/ccCap2.cpp}
\lstinputlisting[caption=C\'odigo XML obtenido\, ejemplo de la disposici\'on b\'asica de un nodo con la llamada a una funci\'on, style=customxml, firstline=98, lastline=104]{Codigos/xmlsCap2.xml}

%-------------------------------------------------------------------
\subsection{Interacci\'on de entrada y salida}
%-------------------------------------------------------------------

En el dise\~no de nuestra herramienta damos mucha importancia a la utilidad que pueda suponer para el usuario final: un programador novel. Es por ello que para que una vez hecho el an\'alisis simb\'olico, en caso de que el usuario quisiera reconstruir el flujo de ejecuci\'on del programa, una utilidad que simule el comportamiento interactivo de una consola puede convertirse en una buena herramienta de depuraci\'on.

A lo largo de la existencia de C y sus posteriores revisiones han idos surgiendo numerosas bibliotecas que se encargan de encapsular la entrada y salida para agilizar la tarea del programador. Entre estas bibliotecas destacan la biblioteca est\'andar de C++ y la biblioteca est\'andar de C.

La biblioteca est\'andar de C (libc) [23] es una recopilaci\'on de ficheros cabecera y bibliotecas con rutinas, estandarizados por un comit\'e de la Organizaci\'on Internacional para la Estandarizaci\'on (ISO). Estas rutinas implementan las operaciones comunes del lenguaje de forma que todos los programas implementados en C se basan en ella para funcionar. Dentro de los ficheros cabecera de la biblioteca el fichero stdio.h contiene las definiciones necesarias para las operaciones de entrada y salida.

La biblioteca est\'andar de C++ (Standard Template Lybrary, STL) [24] incluye una colecci\'on de funciones y clases definidas en el n\'ucleo del lenguaje y est\'a ideada para dar soporte a la mayor\'ia de funcionalidades del lenguaje. Iostream (acr\'onimo de Input/Output Stream) es el componente responsable del control de flujo de entrada y salida y refiere a un conjunto de plantillas de clases que manejan las capacidades de entrada y salida de C++ por medio de flujos (streams). Todos los objetos derivados de iostream son parte del espacio de nombres (namespace) std.

Para simplificar el acceso a las operaciones de entrada y salida teniendo en cuenta su necesaria adaptaci\'on a un formato XML y su posterior procesado por el int\'erprete simb\'olico hemos definido dos sencillas bibliotecas que eliminan las diferencias entre las dos bibliotecas m\'as generalizadas de la programaci\'on en C++ aportando la funcionalidad b\'asica que pueda necesitar el usuario: introducci\'on y emisi\'on de valores enteros o strings.

Las bibliotecas implementadas son BuiltinsSTD.h y BuiltinsIO.h basadas en la entrada y salida de la biblioteca est\'andar de C++ y C respectivamente.

La figura 2.25 muestra las funciones incluidas por las bibliotecas.

\lstinputlisting[caption=Figura ejemplo con las funciones incluidas en las bibliotecas., style=customxml, firstline=106, lastline=108]{Codigos/xmlsCap2.xml}

%-------------------------------------------------------------------
\section{Tecnolog\'ia adicional}
%-------------------------------------------------------------------

Como se ha explicado previamente en nuestra herramienta un factor clave es la producci\'on de un fichero XML equivalente al c\'odigo introducido con la finalidad de realizar la ejecuci\'on simb\'olica a partir de \'el. Desde un principio se descart\'o la posibilidad de implementar una herramienta por nuestra cuenta con las capacidades requeridas, como opci\'on final nos decidimos por emplear alguna de ya existente cuya utilizaci\'on nos estuviera permitida, optamos por la librer\'ia TinyXML2.

%-------------------------------------------------------------------
\subsection*{TinyXML2}
%-------------------------------------------------------------------

TinyXML2 es una librer\'ia de software libre creada para la manipulaci\'on de ficheros XML. TinyXML2 est\'a sujeta a los t\'erminos de la licencia zlib de distribuci\'on de software.

Elegimos TinyXML2 como una opci\'on viable ya que como indica su nombre es una herramienta de tama\~no reducido y de caracter\'isticas ajustadas a la eficiencia que busc\'abamos. En el proceso de generar el fichero XML esta librer\'ia nos ofreci\'o directivas bastante sencillas, as\'i como una manipulaci\'on directa y \'agil de los nodos XML. [25]

A grandes razgos la generaci\'on del fichero que devuelve AST2XML comienza con la creaci\'on de un ``documento en blanco'' y la generaci\'on de un puntero auxiliar. El documento guarda la funcionalidad relativa a la generaci\'on de nuevas estructuras XML y es el que finalmente gestiona el volcado de informaci\'on al fichero resultante. El puntero auxiliar se utiliza como almacenamiento temporal de los nodos XML una vez estos ya han sido finalizados.

%-------------------------------------------------------------------
\section{Uso de AST2XML}
%-------------------------------------------------------------------

Una vez compilado la herramienta, se puede ejecutar esta mediante el comando:

\lstinputlisting[caption=Uso de AST2XML., style=customxml, firstline=110, lastline=110]{Codigos/xmlsCap2.xml}

El primer y el tercer par\'ametro que se emplean en la llamada representan al fichero de c\'odigo del que se quiere obtener una representaci\'on en XML y el fichero resultante.

El segundo par\'ametro es una directiva de la ejecuci\'on propia de herramientas de LLVM que indica la inexistencia de una base de datos de compilaci\'on que podr\'ia ser necesaria al hacer ejecutar otra herramienta. [26]

%-------------------------------------------------------------------
%section*{\NotasBibliograficas}
%-------------------------------------------------------------------


%-------------------------------------------------------------------
%\section*{\NotasBibliograficas}
%-------------------------------------------------------------------
%\TocNotasBibliograficas

%Citamos algo para que aparezca en la bibliograf\'ia...
%\citep{ldesc2e}

%\medskip

%Y tambi\'en ponemos el acr\'onimo \ac{CVS} para que no cruja.

%Ten en cuenta que si no quieres acr\'onimos (o no quieres que te falle la compilaci\'on en ``release'' mientras no tengas ninguno) basta con que no definas la constante \verb+\acronimosEnRelease+ (en \texttt{config.tex}).


%-------------------------------------------------------------------
%\section*{\ProximoCapitulo}
%-------------------------------------------------------------------
%\TocProximoCapitulo

%...

% Variable local para emacs, para  que encuentre el fichero maestro de
% compilaci\'on y funcionen mejor algunas teclas r\'apidas de AucTeX
%%%
%%% Local Variables:
%%% mode: latex
%%% TeX-master: "../Tesis.tex"
%%% End:
