%---------------------------------------------------------------------
%
%                          Apéndice 1
%
%---------------------------------------------------------------------

\chapter{AST2XML's Readme}

\pagestyle{plain}

A simple C++ parser that returns a simplified version of Clang's AST in XML format.

\section{Translation}

\subsection*{Function declaration}

Contains: the name of the function, the return type, the line, a list of parameters required and the body of the function.

        int foo(int a,int b, ... ){
            ...
        }
 ---
        <function name="foo" type="int" line="3">
            <params>
                <param type="int" name="a"/>
                <param type="int" name="b"/>
                ...
            </params>
            <body>
                ...
            </body>
        </function>

\subsection*{Variable declaration}

Contains: the variable type, line and the initialization expresion.

        int a,b;
        int c = 1;
---
        <declarations>
            <declaration type="int" name="a" line="3"/>
            <declaration type="int" name="b" line="3"/>
        </declarations>
        <declarations>
            <declaration type="int" name="c" line="4">
                <const value="1"/>
            </declaration>
        </declarations>

\subsection*{Assignments & Operators}

Contains: the name of the recipient, the expresion to be assigned and the code position. Also the type of the operation and the operator type if needed.

Operators supported are:

        + , - , / , * , ++ , -- , += , -= , /= , *= , && , || , ! , +(sign) , -(sign)
---
        a = 1;
        b += a;
        c++;
        d = -c;
        e = !f;
---
        <assignment name="a" line="72">
            <const value="1"/>
        </assignment>
        <assignmentOperator name="b" type="arithmetic" operator="+" line="73">
            <variable name="a"/>
        </assignmentOperator>
        <unaryOperator name="c" operator="+" line="74"/>
        <assignment name="d" line="75">
            <signOperator type="-">
                <variable name="c"/>
            </signOperator>
        </assignment>
        <assignment name="e" line="76">
            <notOperator>
                <variable name="f"/>
            </notOperator>
        </assignment>


\subsection*{If}

Contains: the condition expresion, the **then** and/or **else** structure and the code position.

        if(a==0){
            ...
        }else{
            ...
        }
---
        <if line="6">
            <binaryOperator type="comparison" operator="==">
                <variable name="a"/>
                <const value="0"/>
            </binaryOperator>
            <then>
                <body>
                    ...
                </body>
            </then>
            <else>
                <body>
                    ...
                </body>
            </else>
        </if>

\subsection*{While loop}

Contains: the condition expresion, the body of the loop and the code position.

        while(a>0){
            ...
        }
---
        <while line="15">
            <binaryOperator type="comparison" operator="&gt;">
                <variable name="a"/>
                <const value="0"/>
            </binaryOperator>
            <body>
                ...
            </body>
        </while>

\subsection*{For loop}

Contains: the code position, the initialization of the control variable, the control variable's condition and step and the loop's body. 

        for(int i=0;i<10;i++){
            ...
        }
---
        <for line="135">
            <declarations>
                <declaration type="int" name="i" line="135">
                    <const value="0"/>
                </declaration>
            </declarations>
            <binaryOperator type="comparison" operator="&lt;">
                <variable name="i"/>
                <const value="10"/>
            </binaryOperator>
            <unaryOperator name="i" operator="+" line="135"/>
            <body>
                ...
            </body>
        </for>

\subsection*{Calls}

Functions behave like expresions. 
Contains: the name of the function, the type and the arguments required.

        a = foo(b,c);
---
        <assignment name="a" line="94">
            <callFunction name="foo" type="int">
                <arg>
                    <variable name="b"/>
                </arg>
                <arg>
                    <variable name="c"/>
                </arg>
            </callFunction>
        </assignment>

\subsection*{Builtins}

In order to work with the project 
        https://github.com/si1314/PFC 
there are included some builtin functions to mask i/o behaviours. Those are the ones included in the files **BuiltinsIO.h** and **BuiltinsSTD.h**

        ConsoleOut_Int(a);
        return ConsoleIn_Int();
---
        <consoleOut line="83">
            <variable name="a"/>
        </consoleOut>
        <return line="79">
            <consoleIn type="int"/>
        </return>

\section{Installation}

Ideally AST2XML would work on any system. But we are still working on it.

\subsection*{Linux}

This has been tested only on Ubuntu 13.

1. Proceed with Clang's build install as stated here:

        http://clang.llvm.org/get_started.html

2. Go to your llvm directory 

        $cd (your llvm directory)/tools/clang/tools

3. Clone the repository at a custom folder 

        $git clone https://github.com/Si1314/AST2XML.git ast2XMLtool

4. Copy the Makefile

5. Go to your build directory 
        
        cd (your build directory)/tools/clang/tools

6. Make a specific directory for the tool (with the same name) 

        $mkdir astXMLtool

7. Paste in it the Makefile

8. Go into the directory 

        $cd astXMLtool

8. Execute make command 

        $make

The tool will be compiled and linked as long as there are no issues

9. Go back to your build directory 

        $cd (your build directory)/Debug + Asserts/build

There should be the tool. As well as the other tools provided in the clang package.

\section{User manual}

In order to execute the tool use this command:

        $./ast2xml <C++file> -- <XMLfile>