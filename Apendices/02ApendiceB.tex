%---------------------------------------------------------------------
%
%                          Apéndice 1
%
%---------------------------------------------------------------------

\chapter{Gram\'atica reducida C++}

En \'este ap\'endice mostramos la gram\'atica de C++ reducida, con las reglas que utiliza nuestra herramienta.

%-------------------------------------------------------------------
\section{Reglas gram\'aticales}
%-------------------------------------------------------------------

\begin{itemize}
\item ElementosEntrada : ElementoEntrada
\item ElementoEntrada : whitespace | comentario | token
\item comentario : ``/'' ``'' caracter\_ASCII - \{ ``/'' \} `` '' ``/'' | ``/'' ``/'' caracter\_ASCII - \{ ``\\n'' \}
\item digito : 0 | 1 | 2 |3 | 4 | 5 | 6 | 7 | 8 | 9
\item letra : a | b | c | d | e | f | g | h | i | j | k | l | m | n | \~n | o | p | q | r | s | t | u | v | w | x | y | z | A | B | C | D | E | F | G | H | I | J | K | L | M | N | \~N | O | P | Q | R | S | T | U | V | W | X | Y | Z | \'A | \'E | \'I | \'O | \'U
\item whitespace : espacio en blanco
\item ascii Caracteres del est\'andar ASCII
\item caracter\_ASCII : ascii | \~N | \~n
\item token : identificador | palabra\_reservada | literal | delimitador | operador
\item identificador : ``'' [ letra | d\'igito | ``'' | ``\$'' ]* letra [ letra | d\'igito | ``'' | ``\$'' ]* ``\$'' [ letra | d\'igito | ``'' | ``\$'' ]*
\item palabra\_reservada : ``class'' | ``else'' | ``for'' | ``if'' | ``int'' | ``return'' | ``void'' | ``while''
\end{itemize}

%-------------------------------------------------------------------
\section{Literales y operadores}
%-------------------------------------------------------------------

\begin{itemize}
\item literal : literal\_integer
\item literal\_integer : [``+'' | ``-''] [digito - \{0\}]* digito* | [``+'' | ``-''] ``0''
\item delimitador : ( | ) | \{ | \} | ; | , | .
\item operador : operadorInfijo | operadorPrefijo | operadorPostfijo
\item operadorInfijo : || | \&\& | == | != | < | > | <= | >= | + | - | * | /
\item operadorPrefijo : ! | + | -
\item operadorPostfijo : ++ | --
\item operadorAsignaci\'on : = | += | -= | *= | /=
\item operadorUnario : \& | * | + | \- | !
\end{itemize}

%-------------------------------------------------------------------
\section{Tipos de datos}
%-------------------------------------------------------------------

\begin{itemize}
\item tipo : tipo\_primitivo identificador
\item tipo\_primitivo : ``int''
\item listaArgumentos : expresionAsignacion | listaArgumentos , expresionAsignacion
\end{itemize}

%-------------------------------------------------------------------
\section{Expresiones}
%-------------------------------------------------------------------

\begin{itemize}
\item expresi\'on : expresionAsignacion | expresion , expresionAsignacion
\item expresionAsignaci\'on : expresionCondicional | expresionUnaria operadorAsignacion expresionAsignacion
\item expresionCondicional : expresionOrLogico
\item expresionOrLogico : expresionAndLogico | expresionOrLogico || expresionAndLogico
\item expresionAndLogico : expresionIgualdad | expresionAndLogico \&\& expresionIgualdad
\item expresionIgualdad : expresionRelacional | expresionIgualdad == expresionRelacional | expresionIgualdad != expresionRelacional
\item expresionRelacional : expresionAditiva | expresionRelacional < expresionAditiva |  expresionRelacional > expresionAditiva | expresionRelacional <= expresionAditiva | expresionRelacional >= expresionAditiva
\item expresionAditiva : expresionMultiplicativa | expresionAditiva + expresionMultiplicativa | expresionAditiva \- expresionMultiplicativa
\item expresionMultiplicativa : expresionConversion | expresionMultiplicativa * expresionConversion | expresionMultiplicativa / expresionConversion
\item expresionConversion : expresionUnaria | ( nombreTipo ) expresionConversion
\item expresionUnaria : expresionSufijo | ++ expresionUnaria | operadorUnario expresionConversion
\item expresionSufijo : expresionPrimaria | expresionSufijo [ expresion ] | expresionSufijo ( listaArgumentos? ) | expresionSufijo ++
\item expresionPrimaria : identificador | digito | literalCadena | ( expresion )
\end{itemize}

%-------------------------------------------------------------------
\section{Expresiones constantes}
%-------------------------------------------------------------------

\begin{itemize}
\item expresionConstante : expresionCondicional
\end{itemize}

%-------------------------------------------------------------------
\section{Sentencias}
%-------------------------------------------------------------------

\begin{itemize}
\item sentencia : sentenciaCompuesta | sentenciaExpresion | sentenciaSeleccion | sentenciaIteracion | sentenciaSalto
\item default : sentencia
\item sentenciaCompuesta : \{ listaDeclaraciones? listaSentencias? \}
\item listaDeclaraciones : declaracion | listaDeclaraciones declaracion
\item listaSentencias : sentencia | listaSentencias sentencia
\item sentenciaExpresion : expresion? ;
\item sentenciaSeleccion : if ( expresion ) sentencia | if ( expresion ) sentencia else sentencia
\item sentenciaIteracion : while ( expresion ) sentencia | for ( expresion? ; expresion? ; expresion? ) sentencia
\item sentenciaSalto : return expresion? ;
\end{itemize}
