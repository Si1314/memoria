%---------------------------------------------------------------------
%
%                          Apéndice C
%
%---------------------------------------------------------------------

\chapter{Readme del Intérprete simbólico en Prolog de Árboles de Sintaxis Abstracta obenidos a partir de código en C++}

\pagestyle{plain}

Este proyecto consiste en un intérprete simbólico que a partir de la representación en XML de un código en C++ realiza una ejecución simbólica de éste.

\section{Funcionamiento}

El interprete recorre la representación del código de una función asignando valores simbólicos a una tabla de variables. Cuando ya ha recorrido el código razona sobre dichos valores y devuelve un conjunto de posibles entradas, posibles salidas, interacciones por consola y la traza de dicha función. Dichos valores resultantes se recogen en un fichero XML.

La ejecución se controla mediante las variables **Inf**, **Sup** y **MaxDepth**. 
Inf y Sup acotan inferior y superiormente el rango que pueden adoptar los valores enteros de las variables.
MaxDepth restringe el número de iteraciones que puede ejecutarse un bucle en la simulación. 

\section{Uso}

Para ejecutar el interprete se invoca el predicado interpreter. Puede haber dos modalidades:

        interpreter(EntryFile, OutFile, Inf, Sup, MaxDepth, FunctionName).
Donde se puede simular con valores específicos en las variables Inf, Sup y MaxDepth.

        interpreter(EntryFile, OutFile, FunctionName).

Con valores predeterminados **Inf=-5**, **Sup=5** y **MaxDepth=10**.

\section{La representación en XML del código}

Este proyecto va de la mano del proyecto AST2XML.

        https://github.com/si1314/AST2XML

El proyecto AST2XML se encarga de traducir el código fuente en una estructura basada en XML equivalente. De esta forma el código:

        int dameMayor(int a, int b){
            if(a > b){
                return a;
            }else{
                return b;
            }
        }

se convierte en:

        <function name="dameMayor" type="int" line="3">
            <params>
                <param type="int" name="a"/>
                <param type="int" name="b"/>
            </params>
            <body>
                <if line="4">
                    <binaryOperator type="comparison" operator="&gt;">
                        <variable name="a"/>
                        <variable name="b"/>
                    </binaryOperator>
                    <then>
                        <body>
                            <return line="5">
                                <variable name="a"/>
                            </return>
                        </body>
                    </then>
                    <else>
                        <body>
                            <return line="7">
                                <variable name="b"/>
                            </return>
                        </body>
                    </else>
                </if>
            </body>
        </function>

\section{Los resultados devueltos}

        <caso>
          <variable name="ret" value="-4"/>
          <variable name="a" value="-4"/>
          <variable name="b" value="-5"/>
          <data>
            <traza> 3 4 5</traza>
            <cin/>
            <cout/>
          </data>
        </caso>
        <caso>
          <variable name="ret" value="-5"/>
          <variable name="a" value="-5"/>
          <variable name="b" value="-5"/>
          <data>
            <traza> 3 4 7</traza>
            <cin/>
            <cout/>
          </data>
        </caso>

\section{Librerías usadas}

\subsection*{CLPFD}

La librería **C**onstraint **L**ogic **P**rogramming over **F**inite **D**omains se emplea para el modelado de la simulación.
Mediante su sistema de restriccioens y asignación de valores simbólicos a las variables lógicas modelamos la ejecución simbólica del programa introducido.

\subsection*{SGML & SGML_WRITE}

Estas librerías adaptadas a la manipulación de ficheros que emplean el sistema **S**tandard **G**eneralized **M**arkup **L**anguage. 
Se emplean en el proyecto a la hora de gestionar la apertura de los documentos XML que contienen el código representado y en la escritura del fichero devuelto.

\section{Librerías propias}

Librerías de creación propia para gestionar la ejecución simbólica de las instrucciones y el cálculo de las expresiones.

\subsection*{Executers.pl}

Las instrucciones se simulan una a una consumiendo la lista de instrucciones que se obtiene a partir del fichero de entrada mediante el predicado **execute**. Este predicado es recursivo, de forma que cada instrucción en algun momento tiene la forma:

        execute(EntryS,[Instruccion|RestoDeInstrucciones],OutS):-
            
            ...
            
            execute(EntryS1,RestoDeInstrucciones,OutS).

\subsection*{Expressions.pl}

Las expresiones se calculan de forma similar:

        resolveExpression(EntryS,('NombreDeLaOperacion',Operador,[ExpresionA,ExpresionB]),Resultado,OutS):-

            ...

            work(Operador,OperandoA,OperandoB,Result).

Siendo en nuestra nomenclatura EntryS y OutS los estados previo y posterior a la simulación de la instrucción o expresión.

\subsection*{AuxiliaryFunctions.pl}

En esta librería están presentes las distintas funciones auxiliares que ayudan a hacer más comprensible el código de las librerías Expressions y Executers.
Destaca la implementación del estado en el que se apoyan dichas librerías.

En cada ejecución del interprete simulamos la fluctuación de los valores de un conjunto de variables. 
Estas variables las manejamos mediante la tupla **state**.

        state = (Tabla,Consola,Trace)

Esta tupla contiene una tabla variable-valor, una lista de interacciones de consola simuladas y una lista de las líneas ejecutadas del código.

\subsection*{VariablesTable.pl}

En esta librería están aglutinadas las diferentes funcionalidades que soporta la tabla de variables-valor empleada en la ejecución simbólica de los programas.
