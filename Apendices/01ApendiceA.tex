%---------------------------------------------------------------------
%
%                          Ap�ndice 1
%
%---------------------------------------------------------------------

\chapter{Manual de usuario}

%-------------------------------------------------------------------
\section{Requisitos previos}
%-------------------------------------------------------------------

Como ya hemos indicado en las secciones de la memoria, nuestro proyecto actualmente s�lo funciona en sistemas operativos LINUX, por tanto en lo siguiente supondremos que se trabaja bajo este sistema operativo.

Por otro lado para poder ejecutar el sistema que hemos desarrollado ser� necesario cumplir una serie de requisitos software previos. En concreto deberemos tener instalados los siguientes componentes:

\begin{itemize}
\item \textbf{''Java Runtime Environment''} o JRE de 32 bits, versi�n 1.6 o superior.\\
Para ello desde la propia p�gina recomiendan, en lo relativo a requisitos hardware, contar con un Pentium 2 a 266 MHz o un procesador m�s r�pido con al menos 128 MB de RAM f�sica.\\
Enlace de descarga: 
\url{http://www.java.com/es/download/}
\item \textbf{''Clang''}. Al comienzo del proyecto la �ltima versi�n disponible era la 3.5, por lo que esta fue la que se utiliz�. \\
Para poder compilar nuestra herramienta de generaci�n del AST2XML, cuyos pasos detallados de instalaci�n comentamos en la pr�xima secci�n, es necesario tener instalado Clang en nuestro sistema. Puesto que la instalaci�n y requisitos ya est�n detallados en la propia web de Clang no vamos a detenernos a comentarlos uno a uno.
\begin {itemize}
\item \textbf{''Requisitos LLVM System''}: \\
Enlace de requisitos:
\url{http://llvm.org/docs/GettingStarted.html#requirements}
\item \textbf{''Python''}. Para compilar Clang es necesario tener instalado Python en nuestro sistema. \\
Enlace de descarga:
\url{https://www.python.org/download}
\item \textbf{''Compilacion de Clang''} \\
Enlace con los pasos detallados
\url {http://clang.llvm.org/get_started.html}
\end {itemize} 
\end {itemize}


%-------------------------------------------------------------------
\section{Instalaci�n herramienta ast2xml}
%-------------------------------------------------------------------

La herramienta ast2xml es necesaria compilarla en nuestro ordenador para poder utilizarla, puesto que es diferente para cada equipo. En esta secci�n damos por sentado que el usuario ya ha instalado Clang en su ordenador, a continuaci�n detallamos los pasos para su compilaci�n en nuestro sistema para luego poder utilizarla.

\begin{itemize}
\item Dirigirse a nuestro directorio LLVM. \\
\textbf{cd (directorio llvm)/tools/clang/tools}
\item Clonar el repositorio con nuestra herramienta en una carpeta del directorio.\\
\textbf{git clone \url{https://github.com/Si1314/AST2XML.git} ast2xmltool}
\item Copiar el Makefile.
\item Ir al directorio de compilaci�n.\\
\textbf{cd (directorio de compilaci�n)/tools/clang/tools}
\item Crear un directorio espec�fico para la herramienta (con el mismo nombre).\\
\textbf{mkdir ast2xmltool}
\item Copiar en �l el archivo Makefile.
\item Ir dentro del directorio.\\
\textbf{cd ast2xmltool}
\item Ejecutar el comando ``make''\\
La herramienta estar� compilada y enlazada si no hay ning�n problema
\item Volver al directorio de compilaci�n.\\
\textbf{cd (directorio de compilaci�n)/Debug + Asserts/build}

Aqu� deber�a encontrarse la herramienta, as� como el resto de herramientas proporcionadas por el paquete de clang.
\end{itemize}

%-------------------------------------------------------------------
\section{Tutorial de uso}
%-------------------------------------------------------------------

 
% Variable local para emacs, para  que encuentre el fichero maestro de
% compilaci�n y funcionen mejor algunas teclas r�pidas de AucTeX
%%%
%%% Local Variables:
%%% mode: latex
%%% TeX-master: "../Tesis.tex"
%%% End:
