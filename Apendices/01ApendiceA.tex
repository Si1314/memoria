%---------------------------------------------------------------------
%
%                          Ap�ndice 1
%
%---------------------------------------------------------------------

\chapter{Manual de usuario}

%-------------------------------------------------------------------
\section{Requisitos previos}
%-------------------------------------------------------------------
\label{ap1:intro}

Como ya hemos indicado en las secciones de la memoria, nuestro proyecto actualmente s�lo funciona en sistemas operativos LINUX, por tanto en lo siguiente supondremos que se trabaja bajo este sistema operativo.

Por otro lado para poder ejecutar el sistema que hemos desarrollado ser� necesario cumplir una serie de requisitos software previos. En concreto deberemos tener instalados los siguientes componentes:

\begin{itemize}
\item Java Runtime Environment o JRE de 32 bits, versi�n 1.6 o superior.\\
Para ello desde la propia p�gina recomiendan, en lo relativo a requisitos hardware, contar con un Pentium 2 a 266 MHz o un procesador m�s r�pido con al menos 128 MB de RAM f�sica.\\
Enlace de descarga: 
\url{http://www.java.com/es/download/}
\item Clang. Al comienzo del proyecto la �ltima versi�n disponible era la 3.5, por lo que esta fue la que se utiliz�. \\
Para poder compilar nuestra herramienta de generaci�n del AST2XML, cuyos pasos detallados de instalaci�n comentamos en la pr�xima secci�n, es necesario tener instalado Clang en nuestro sistema. Puesto que la instalaci�n y requisitos ya est�n detallados en la propia web de Clang no vamos a detenernos a comentarlos uno a uno.
\begin {itemize}
\item Requisitos LLVM System \\
Enlace de requisitos:
\url{http://llvm.org/docs/GettingStarted.html#requirements}
\item Python. Para compilar Clang es necesario tener instalado Python en nuestro sistema. \\
Enlace de descarga:
\url{https://www.python.org/download}
\item Compilacion de Clang \\
Enlace con los pasos detallados
\url {http://clang.llvm.org/get_started.html}
\end {itemize} 

\end {itemize}


%-------------------------------------------------------------------
\section{Instalaci�n Clang}
%-------------------------------------------------------------------
\label{ap2:clang}

 
% Variable local para emacs, para  que encuentre el fichero maestro de
% compilaci�n y funcionen mejor algunas teclas r�pidas de AucTeX
%%%
%%% Local Variables:
%%% mode: latex
%%% TeX-master: "../Tesis.tex"
%%% End:
